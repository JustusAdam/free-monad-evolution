\section{Conclusion}

This paper presents the conception and development of extensible effect systems.
It shows how a monad capable of handling extensible sets of effects can be
constructed by decoupling the effect from the monad using a free monad and
replacing concrete effects with an extensible union type. The implementation is
reviewed and optimised to attain competitive performance with transformers and a
minimal interface for effects is found resulting in several improvements over
current solutions:

\begin{itemize}
  \item Defining new effects involves minimal boilerplate. Necessary components
    are only an encoding data type and an interpreter.
  \item Computations using effects are constrained using the same member
    constraint regardless of effect, relieving the need for effect classes and
    instances.
  \item The effect monad is much more transparent, enabling combinators for
    effect relaying and generic handling.
\end{itemize}

Other, similar, solutions are examined and compared, which suggest that even
more powerful effect systems can be built upon these foundations and that
extensible effects may be a safer, more expressive alternative to monolithic
monads such as IO.
