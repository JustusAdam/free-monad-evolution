\section{Conclusion}

Extensible effect systems have enjoyed a significant amount of attention by the
scientific community in recent years. They provide means to write code which is
very expressive at the type level via fine grained compositions of effects and
even improved type safety via resource tracking through the environment. Since
such computations are expressed in data structures rather than opaque functions
the can be inspected, altered, filtered or otherwise interpreted depending on
context and requirements. This aids reusability of the code on one hand, and
simplifies mocking and testing on the other. It also makes it easier to refactor
code and change implementations, because the functions using the effects are
implemented against abstract interfaces rather than concrete effects.
The PureScript language is an example of a language that already employs
an extensible effects system to structure its computations.

Extensible effects may finally provide a solution to the problem of monolithic
monads such as Haskell's \texttt{IO}.

A popular solution is the effectful, extensible monad based on the \emph{free
  monad} which has the desired characteristics when combined with an open union,
or equivalent structure.

Thanks to additional effort being put into improving the performance of
constructing a free monad and reflecting on the computations, extensible effect
systems begin to outperform traditional systems such as MTL.

Regardless of the concrete implementation extensible effect systems have
certainly succeeded in simplifying the definition of monadic effects as well as
means to test, mock and reuse them. Extensible effect systems have certainly
come to stay.
