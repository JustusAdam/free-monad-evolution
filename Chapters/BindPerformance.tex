\section{Asymptotic performance of $\bindOp$}

\label{sec:bind-performance}

As mentioned before the asymptotic performance of $\bindOp$ for the
\texttt{Free} monad is not good. As can be seen in figure~\ref{fig:free-monad}
the implementation for $\bindOp$ pushes the $\bindOp$ down the tree using
\texttt{fmap} until it reaches a \texttt{Pure} value. This means that for every
use of $\bindOp$ the entire tree must be traversed once.

\subsection{Better asymptotic performance using continuation passing}

\label{sec:performance-with-codensity}

This problem is akin to appending to singely linked lists where the entire list
has to be traversed to find the final pointer and append. A solution to this
problem was independently developed in 2008 by Janis
Voigtländer~\cite{asymptotic-performance-improvement} in an attempt to make free
monads more efficient. The idea is rather simple and again related to the
problem of list appends. For list appends the solution is the so called
\emph{difference list}~\cite{difference-list} which really is not a list but a
function of type \texttt{[a] -> [a]}. This function will, when called with a
list, append it to its end and return the resulting list. Normal list
operations, such as appending and prepending are realised via function
composition. This is often also described as building the list while leaving a
``hole'' at its end to be plugged with a terminating value once the construction
is complete. One this value is provided, which is usually just an empty list, it
can be built in a single step. In imperative programming this is known as the
\emph{builder pattern}. A hole is left to the end of the list and all altering
operations are realised with function composition.

Asymptotic performance improvement of the free monad employs a similar idea. It
uses a generic structure from category theory, which is now known as the
\emph{Codensity monad}. The original paper by
Voigtländer~\cite{asymptotic-performance-improvement} called it \texttt{C}. The
codensity monad, shown in figure~\ref{fig:codensity-monad}, is also a function
which, given a continuation, builds some monad \texttt{m}. The \emph{extensible
  effects} library, developed by Kiselyov et al~\cite{extensible-effects} uses
this codensity monad, specialised to a variant of \texttt{Free}. The figure also
shows the new implementation for $\bindOp$ which does not push down the tree
anymore, but builds two lambda functions, which are passed as continuations, a
much cheaper operation in Haskell.

\begin{figure}
  \lstinputlisting[firstline=9]{Listings/Codensity.hs}
  \caption{The codensity monad}
  \label{fig:codensity-monad}
\end{figure}

\subsection{Improving interpreter performance using type aligned sequences}

\label{sec:type-aligned-sequence}

The builder pattern, or programming with holes and continuations can drastically
improve the construction of a value. However this only work when the structure
is built in its entirety and then inspected in a second step. If the structure
needs to be read \emph{during} its construction this approach cannot be used.
Because a function is an opaque object, the intermediate values of a list, or in
this case a forming monad chain, cannot be read, necessitating an evaluation of
the structure before reading. Subsequent constructions can be sped up again, but
an intermediate structure was allocated to facilitate the reads, diminishing the
achieved gains.

For the effect system, implemented by our free monads, in particular, this
situation occurs in the interpreter. In order an effect we first must inspect
the computation. When the effect has been handled the continuation has to be
pushed through the monad, necessitating a full traversal of the (potentially
growing) effect chain. The crucial point where this traversal is performed can
be seen in figure~\ref{fig:ee-handle-relay} where the continuation is not simply
appended, as was the case in figure~\ref{fig:codensity-monad}, but traverses the
entire stack, similar to figure~\ref{fig:free-monad}.

\begin{figure}
  \lstinputlisting{Listings/EEHandleRelay.hs}
  \caption{Relay function as implemented in~\cite{extensible-effects}}
  \label{fig:ee-handle-relay}
\end{figure}

The principal problem her is that new continuations, in the form on lambda
functions are created and partial applications are pushed into the functor. This
is a typical case where allocation occurs in Haskell. While these are eventually
reduced to smaller and simpler expressions, the intermediate allocations never
the less occur which increases both the time needed and the space. As the
evaluation from the extensible effects paper~\cite{freer} shows this regression
in time and space only occurs when a \emph{used} effect is threaded through the
computation, meaning the second clause in figure~\ref{fig:ee-handle-relay} is
used. In the case where the reader effect is \emph{under} the state, this clause
is triggered much less often, explaining the linear performance. Such a
performance degradation caused by the order of effects is highly undesirable. It
would be difficult to a user of this system to utilise it correctly in the
presence of such subtle differences influencing the performance.

A recent paper by van der Ploeg and Kielyov~\cite{ftc-queue} designed a solution
to this problem in the form of so called \emph{type aligned sequences}. The
approach is applicable much more broadly than only free monads. Van der Ploeg
and Kiselyov represent chains of computations with data structures via the use
of Generalized Algebraic Data Types or GADTs. Whereas usually chains of
computations are simply composed, instead they are stored in a data structure
that captures how input and return types relates, preserving the type safety. A
small example of one such structure can be seen in
figure~\ref{fig:type-aligned-list}. This affords the possibility to inspect and
alter individual links of the chain. Type safety is preserved via the GADT which
does not allow the storage of an incompatible chain segments. For the free monad
in particular a type aligned double-ended queue is used, represented internally
as a tree. Now the handler computations inspect only the head of the queue and
push new types of effects back onto the front without altering the tail.

\begin{figure}
  \lstinputlisting{Listings/TypeAlignedList.hs}
  \caption{A type aligned list of functions}
  \label{fig:type-aligned-list}
\end{figure}

With the two improvements from these last two sections the computational
complexity of extensible effects improves dramatically. Whereas the naive
implementation from section~\ref{sec:free} worse than quadratic in both time and
space, the improved version performs linearly with respect to both. This is not
only an improvement over the MTL library. According to the performance
evaluation in \cite{freer} the MTL library is \emph{faster} for a single state
effect, due to GHC specific optimisations deliberately targeted at the MTL
\texttt{State} monad. However MTL does not scale well with more effects. It
scales quadratically in both time and space, regardless of the ordering of
effects.
