\section{Improving the performance of $\bindOp$}

\label{sec:bind-performance}

While there is beauty in the simplicity of the bare free monad its performance
is lacking. As can be seen in figure~\ref{fig:free-monad} the implementation for
$\bindOp$ pushes the $\bindOp$ down the tree using \texttt{fmap} until it
reaches a \texttt{Pure} value. This means that for every use of $\bindOp$ the
entire tree must be traversed once.

\subsection{Better asymptotic performance using continuation passing}

\label{sec:performance-with-codensity}

This problem is akin to appending to singely linked lists where the entire list
has to be traversed to find the final pointer and append. \cite{difference-list}
propose a solution whereby an unfinished list is represented as a function
\texttt{[a] -> [a]} which appends a list to the intermediate result thus
retaining a reference to the end of the list. List
altering functions, such as appends and prepends are implemented as function
composition and thus very cheap. When the construction step is completed the
list can be simply realised by passing \texttt{[]} to the function.

\cite{asymptotic-performance-improvement}s
solution~\cite{asymptotic-performance-improvement} an adaptation of the idea,
applied to moands. He uses a \emph{Codensity monad} shown in
figure~\ref{fig:codensity-monad}, which is a function that, given a
continuation, builds some monad \texttt{m}. As the figure shows this $\bindOp$
does not traverse the structure. Analogue to the list solution it retains a
reference to the end.

\begin{figure}
  \lstinputlisting[firstline=9]{Listings/Codensity.hs}
  \caption{The codensity monad}
  \label{fig:codensity-monad}
\end{figure}

\subsection{Improving interpreter performance using type aligned sequences}

\label{sec:type-aligned-sequence}

While a continuation based free monad improves the performance of $\bindOp$ it
only does so during the initial construction. During interpretation the
codensity monad is evaluated to be able to interpret the head of the constructed
free monad. In \texttt{handle\_relay}, when the handler responds to its own
effect, see~\ref{fig:ee-handle-relay}, the constructed free monad is traversed
similarly to $\bindOp$ for \texttt{Free}. As a result handling effects is again
associated with quadratic complexity, as was measured by
\citeauthor{freer}~\cite[§4.1]{freer}.

\begin{figure}
  \lstinputlisting{Listings/EEHandleRelay.hs}
  \caption{Relay function as implemented in~\cite{extensible-effects}}
  \label{fig:ee-handle-relay}
\end{figure}

The principal problem her is that new continuations, in the form on lambda
functions are created and partial applications are pushed into the functor.
These intermediate allocations cause the increase in both time and space
complexity. As the evaluation from the extensible effects paper~\cite{freer}
shows this regression in time and space only occurs when a \emph{used} effect is
threaded through the computation. In the case where the reader effect is
\emph{under} the state, this clause is triggered when only on the last layer,
explaining the linear performance.

A recent paper by van der Ploeg and Kielyov~\cite{ftc-queue} proposes a solution
to this problem in the form of so called \emph{type aligned sequences}. The
approach is applicable much more broadly than just free monads. Van der Ploeg
and Kiselyov represent chains of computations with data structures via the use
of Generalized Algebraic Data Types. Whereas usually chains of computations are
simply composed, instead they are stored in a data structure that captures how
input and return types relates, preserving the type safety. This affords the
possibility to inspect and alter individual links of the chain. Type safety is
preserved via the GADT which does not allow the storage of an incompatible chain
segments. For the free monad in particular a type aligned double-ended queue is
used, represented internally as a tree. Using this sequence handler computations
inspect only the head of the queue and push new types of effects back onto the
front without altering the tail.
