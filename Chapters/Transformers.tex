\section{Current State of the Art}

\label{sec:transformers}

Monad transformers are the current state of the art when it comes to composing
monads in Haskell. Each type of effect is implemented by what
\citeauthor{steele-pseudomonad} called a
\emph{pseudomonad}~\cite{steele-pseudomonad} and \citeauthor{monad-transformers}
later broadened into the concept of a
\emph{transformer}~\cite{monad-transformers}, a structure which leaves a
``hole''~\cite{moggi-hole} in a monad to embed extensions. Concretely these
``incomplete'' monads are parameterised by an additional type variable which is
instantiated with an arbitrary monad, combining the effect of the transformer
and the inner monad. As an example see the \texttt{StateT} monad transformer in
figure~\ref{fig:state-monad} into which an arbitrary monad \texttt{m} is
embedded. To access effects of the inner monad each transformer implements a
\texttt{lift :: (MonadTrans m, Monad n) => n a -> m a} operation to embed
computations using the inner monad in the transformer. Complex effect systems
are built up by nesting multiple transformers into each other, forming a
transformer stack which, is terminated by some none-transformer monad, such as
\texttt{IO} or \texttt{Identity}.

A key idea of the approach of transformers is to abstract away from the concrete
transformer using a type class to define an interface for its primitive
operations~\cite{jones-constructor-classes}, which the associated transformer
implements, such as the example in figure~\ref{fig:state-state-instance}. The
same primitive operation may now be implemented by different monads, and
crucially, other monad transformers. Transformers that do not implement a
particular effect delegate its execution up the stack with \texttt{lift}. Thus
these classes can be used to ``auto lift'' effects up the stack to the layer
capable of handling it.

Furthermore computations using effects ban be implemented in terms of effect
interfaces used, by constraining an abstract monad with effect classes, rather
than concrete stacks or transformers, which makes them reusable in different
monad stacks.

\begin{figure}
  \lstinputlisting[lastline=11]{Listings/State.hs}
  \caption{The \texttt{State} monad transformer}
  \label{fig:state-monad}
\end{figure}

\begin{figure}
  \lstinputlisting[firstline=13]{Listings/State.hs}
  \caption{Monad class for the state effect and its implementation}
  \label{fig:state-state-instance}
\end{figure}

Transformers and classes have two main shortcomings however.

\begin{enumerate}
\item \textbf{Boilerplate}. The class machinery to overload effects requires
  instances of each effect class, for each transformer. Newly defined
  transformers must thus provide a boilerplate instance of every other effect
  class delegating the effects using \texttt{lift}. Similarly newly defined
  effects must also provide an instance for their effect class for
  every preexisting transformer to lift the effects.
  \item \textbf{Performance}. Every monadic action such as \texttt{pure},
    $\bindOp$ and the effects must traverse the entire stack adding overhead
    scaling quadratically with the size of the stack as shown by
    \citeauthor{freer}~\cite[§4.1]{freer}.
\end{enumerate}
