\section{A better open union}

\label{sec:better-open-union}

Monad transformers have been a part of the Haskell ecosystem for a long time.
They are the de-facto standard method used to implement modular effect systems.
Given how useful and necessary, if not unavoidable, effect systems are for
``real world applictions'' it is not surprising that monad transformers are now
widely used in libraries and even more so in applications and frameworks. being
the biggest player in the game and ubiquitously used comes with benefits for a
library. Namely, in this case, the attention of the compiler writers. According
to Kiselyof and Ishii~\cite{freer} the \texttt{State} monad in particular
enjoys ``preferential treatment by GHC'' as the authors put it, ``with dedicated
optimisation passes''. For an alternative effect system to be effective
therefore it must provide at least comparable performance to the \texttt{mtl}
library, it will not see significant adoption.

The free monad in conjunction with the open union from the previous section is
unable to compete with the \texttt{mtl}. Encoding the open union of types with a
linked list is not only slow but also degrades as the number of distinct effects
increases, discouraging modularity and reusability. This section will explore
alternative ways of representing an open union with the aim to improve its
efficiency. The next section will deal with improving the asymptotic performance
of the bind operator $\bindOp$ for free monads.

In general union types, also called algebraic data types, are implemented as
variably sized structures, prefixed by a tag of known size (typically a byte or
machine word) which tells the inspecting party which concrete shape, also called
constructor, inhabits this value.

For a closed union the possible shapes the value can assume are known at compile
time, enabling efficient code generation and statically checked pattern matches.
To implement an extensible effect system however the union cannot be closed.
Neither the concrete types nor their order is known when the effect monad itself
is implemented. Furthermore it is desirable to be able to add additional effects
for subcomputations only. Therefore the union must support dynamically adding
constructors. A key idea from Kiselyov, Sably and
Swords~\cite{extensible-effects} is that open unions can be implemented
similarly to closed unions by coercing the concrete value to an untyped one and
pairing it with a tag to discern its type at runtime. The first published
version of this technique can be seen in figure~\ref{fig:union-type}.

In its first instance the authors used the builtin reflection
mechanism provided by the GHC compiler\cite{ghc}. It generates a runtime
representation of a type called \texttt{TypeRep}. Each such \texttt{TypeRep}
structure carries a compiler generated 128 bit unique MD5 hash of that concrete
type. \texttt{TypeRep}'s are static constants and support fast comparison by
comparing their fingerprints.

While this allows for the implementation of a single-struct open union, as
opposed to the linked list from before, resolving the type still involves
several steps.
\begin{enumerate*}
\item Pointer dereference to get to the \texttt{TypeRep} structure.
\item tag comparison for the two type reps since \texttt{TypeRep} is itself an
  algebraic data type and
\item comparing the two 128 bit values.
\end{enumerate*}
Considering the generally small size the union will have a 128 bit fingerprint
is unnecessarily large.

The followup paper by Kiselyov et al.~\cite{freer} proposes an alternative
source for a tag value, the index in the type level list. Since each union
\texttt{Union r a} carries with it the type level list \texttt{r} each functor
type \texttt{t} to which we can coerce a value must be present in this list and
this have an index that is known at compile time. This index is much smaller
than the fingerprint. The original implementation in~\cite{freer} used an
\texttt{Int} value, which is at least 30 bit, later implementations use
\texttt{Word} which is the same size, but unsigned. Unlike the \texttt{TypeRep}
from before this value can be directly inlined into the \texttt{Union}
constructor, alleviating the need for both the pointer dereference as well as
the tag comparison. As a result matching values of this union type is almost as
efficient as matching a native, closed union value.
