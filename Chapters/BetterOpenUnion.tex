\section{Improving the Open Union}

\label{sec:better-open-union}

For extensible effect systems to become viable it is important that its
performance is comparable with the currently employed solutions.
Monad transformers have been a part of the Haskell ecosystem for a long time and
according to \citeauthor{freer} the \texttt{State} monad in particular benefits
from dedicated optimisation passes in the Haskell compiler GHC.

The free monad in conjunction with the open union from the previous section is
unable to compete with the \texttt{mtl}. Encoding the open union of types with a
chain of sums is inefficient. Its structure is similar to a linked list and
hence it has similar computational complexity.

To construct a less space consuming open union, an encoding similar to the
encoding of closed unions is used. Efficient closed unions pair the payload
with a tag value. At runtime code dispatches based on the value of the tag.

GHC provides a built-in mechanism for generating runtime representations of
types called (\texttt{TypeRep}). \texttt{TypeRep}s support fast comparison via a
compiler generated unique 128 bit MD5 hash. Figure~\ref{fig:union-type} shows
the implementation of the reflection based union.

\begin{figure}
  \lstinputlisting{Listings/EfficientUnion.hs}
  \caption{A single value open union based on reflection}\footnote{Slightly
      simplified and with the \texttt{Member} class renamed for consistency.}
  \label{fig:union-type}
\end{figure}

While this allows for the implementation of a single-struct open union, as
opposed to the linked list from before, resolving the type still involves
several steps.
\begin{enumerate*}
\item Pointer dereference to get to the \texttt{TypeRep} structure.
\item tag comparison for the two type reps since \texttt{TypeRep} is itself an
  algebraic data type and
\item comparing the two 128 bit values.
\end{enumerate*}
Considering the generally small size the union will have a 128 bit fingerprint
is unnecessarily large.

The followup paper by \citeauthor{freer}~\cite{freer} proposes an alternative
source for a tag value, the index in the type level list. Since each union
\texttt{Union r a} carries with it the type level list \texttt{r} each functor
type \texttt{t} to which we can coerce a value must be present in this list and
this have an index that is known at compile time. This index is much smaller
than the fingerprint. The original implementation in~\cite{freer} used an
\texttt{Int} value, which is at least 30 bit, later implementations use
\texttt{Word} which is the same size, but unsigned. Unlike the \texttt{TypeRep}
from before this value can be directly inlined into the \texttt{Union}
constructor, alleviating the need for both the pointer dereference as well as
the tag comparison.

The index based approach also removes the need to derive a \texttt{Typeable}
instance for the effect type.
