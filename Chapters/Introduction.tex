\section{Introduction}

\label{sec:introduction}

Monads are a powerful abstraction to define side effects. Given a sufficiently
capable type system, effectful computations can be expressed as first class
values with regular, higher kinded types. Monadic values however are inherently
opaque, meaning computations only compose within the same monad. As a result
interleaving different effects, that is composing different monads, poses a
challenge.

Monad transformers~\cite{transformer-inspiration} use the approach of ``monads
with a hole'', which employs an unfilled type parameter, representing an
``inner'' monad, to which other effects can be delegated.

While transformers solve the principal composition issue, which will be shown in
more detail in §\ref{sec:transformers}, they involve a large amount of
boilerplate code and the performance degrades quadratically~\cite{freer} with
the number of transformers employed in a given monad.

This paper argues a different approach. Rather than using stacks of monads a
single \emph{extensible monad} is employed. It is based on the \emph{free
  monad}, which captures the essence of what it means to be a Monad, and has no
capabilities of its own. Instead effects are encoded as values from a set of
effects. Computations are expressed as sequences of effect values and
continuations, which are handled by an interpreter. Computations using effects
are implemented in terms of member constraints on the effect set, and thus
reusable in different concrete effect sets and with different interpreters.

Boilerplate code is reduced substantially, as defining new effects involves only
the definition of its encoding values and one or more interpreters.

To implement an extensible effect system first a naive but simple implementation
of a free monad and an effect set is shown (§\ref{sec:free}). Subsequently this
approach is improved to be competitive with previous approaches to make it
viable for general use (§\ref{sec:better-open-union},
§\ref{sec:bind-performance}) and further simplify the interface
(§\ref{sec:freer}) for defining effects. The implemented approach is
contextualised with other, similar systems in terms of both how they are
implemented and the feature set they offer (§\ref{sec:alternative-systems}).
