\section{Relinquishing the Functor Constraint While Gaining Performance}

\label{sec:freer}

Upon inspection a pattern can be noticed in the effect values used in extensible
effect systems. The ``functor'' part of the value i.e. the one being transformed
upon application of \texttt{fmap} tends to be a function of type \texttt{t
  $\rightarrow$ a}, where \texttt{t} is the type of value returned by the effect
and \texttt{a} is the type parameterising the functor. In cases where the effect
returns nothing this field is always type \texttt{a} which could be rewritten to
\texttt{() $\rightarrow$ a} to fit this pattern.

This is a known pattern exploited by the \emph{coyoneda functor}, shown in
figure~\ref{fig:coyoneda}. The coyoneda functor can be thought of like a ``free
functor'' and analogous to the free monad it provides a functor instance for
any type of kind \texttt{* $\rightarrow$ *}.

By exploiting this pattern and combining the coyoneda functor with the free
monad we can construct a ``freer'' monad, which works on any higher kinded
effect value and requires no functor instance.

\begin{figure}
  \lstinputlisting[firstline=3]{Listings/Coyoneda.hs}
  \caption{The coyoneda functor}
  \label{fig:coyoneda}
\end{figure}

\begin{figure}
  \lstinputlisting[firstline=3]{Listings/Freer.hs}
  \caption{The augmented free monad with decoupled effect}
  \label{fig:freer-monad}
\end{figure}

Figure~\ref{fig:freer-monad} shows the augmented free monad with the
\texttt{Impure} case now containing \emph{two} fields, one for the effect used
and one for its continuation as can be seen in figure~\ref{fig:freer-monad}.

Decoupling effect and continuation this way is what enables the use of the type
aligned sequence in section~\ref{sec:type-aligned-sequence}. Since the
continuation is no longer hidden in the effect functor and has a the general
structure of \texttt{a $\rightarrow$ Eff eff b} it no longer needs to be an
actual function, but can be replaced by something which can be used \emph{like a
  function}, in case of the freer monad a sequence of continuations.

Uncoupling effect and continuation results in a much more expressive interface
for effects. Whereas before each effect contained some additional fields for
continuations, now their mere signatures concisely express \emph{what} the
effect actually does. Freer effects use GADTs to express which forms an effect
may assume. Thus an effect is associated with some potential inputs, stored as
fields in the effect value and later available in the effect interpreter and the
type of data returned from the effect to the computation via the type variable
parameterising the effect type. For instance in
figure~\ref{fig:console-io-freer} is the console effect from
section~\ref{sec:free} and figure~\ref{fig:console-io-example}. The two
constructors of the effect now describe, in types, what to expect, similar to a
type signature in traditional effectful computations. \texttt{ReadLine} takes no
input and returns a string, indicating that this effect will fetch some
\texttt{String} value \emph{from} the environment, whereas the
\texttt{WriteLine} effect receives a \texttt{String} as input and returns the
unit value, meaning no output is produced, which indicates that a
\texttt{String} value is delivered \emph{to} the environment.

\begin{figure}
  \lstinputlisting[firstline=4]{Listings/ConsoleIOFreer.hs}
  \caption{Freer version of the console IO effect}
  \label{fig:console-io-freer}
\end{figure}

Lastly the handler for an effect no longer has to care about the continuation.
Effect handlers are, in their simplest form, functions of type \texttt{f a $\rightarrow$
  Eff effs a}, where \texttt{f} is the effect type, for instance
\texttt{ConsoleIO}, and \texttt{effs} does not contain \texttt{f} anymore. An
example, how simple such a handler may look, can also be seen in
figure~\ref{fig:console-io-freer}.

With the \texttt{Typeable} requirement gone (see
section~\ref{sec:better-open-union}) as well as the \texttt{Functor} constraint,
expressive effect signatures and simple handlers, the interface for defining and
handling effects becomes small and easy to implement. Complicated effects can be
remapped to standard effects which can be handled generically, such as the
\texttt{State}, \texttt{Reader} or \texttt{Error} effect. The combination of
these features mean that effects are quick and easy to define. This makes it
feasible to implement systems with a much smaller finer effect granularity,
which in turn makes the effect reusable, easier to reason about and test. An
important contributing factor for the viability of this finely granular approach
is the improved performance characteristics as described in the last section.
Due to the linear scaling of freer monads finely granular effect systems can be
employed without having to expect massive performance penalty as would be the
case when using MTL.

An additional feature of the interface of the two systems described in this and
the previous section is that multiple of the same effect can be combined in the
same computation. Namely parameterised effects, such as \texttt{State s}, which
is parameterised over the concrete state type \texttt{s} can occur multiple
times in the effect set, so long as it handles a different \texttt{s}. Thus a
computation of type \texttt{Eff [State Int, State String] a} is possible and
will compute as would be expected. If multiple of identically instantiated
effects are present, for instance \texttt{Eff [State Int, State Int] a}, the
effect closer to the head of the list is handled and used first, then the
latter. This is seldom used in an actual computation, but may occur due to a
handler pushing a new state layer onto the computation, which is subsequently
handles. This type of delegation is highly useful when dealing with complex
custom effects and thus a feature. One slight problem with allowing multiple
types of the same effect is that it becomes more common to have to annotate
polymorphic functions such as \texttt{put} and \texttt{get} with concrete types
to disambiguate the used state type, even if according to the constraints of the
function using it, only one type of state is in scope. However it does afford
finer granularity of effects and composability. Furthermore since defining
custom effects that need no disambiguation is so simple with these systems, it
should not pose problematic.
