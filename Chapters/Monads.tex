\section{Monads}

\label{sec:monads}

Purity, often also called referential transparency, is the property of
a function to always produce the same result when called with the same
arguments. It is a desirable property as it makes it considerably
easier to reason about the behaviour of a program, particularly with
respects to refactoring. But it also allows optimisations, such as
common subexpression elimination and memoization.

However, many tasks that we wish programs to perform cannot be
expressed in terms of pure functions as they entail an interaction
with the world outside of the program, such as accessing a database or
querying the file system. Even inside of the program it is often
useful to define some ambient environment in which certain tasks are
performed, from this object oriented programming was conceived where
each function, or method, carries around the implicit environment of
an object.

To perform the aforementioned tasks many programming languages opt to sacrifice
purity and allow arbitrary side effects in any function. The designers of the
Haskell programming language however found a way to enable side effects in
programs without sacrificing purity. \textbf{Monads}, in the Haskell sense, are
a class of types that describe some sort of environment, which can be interacted
with. The concrete interactions a particular monad allows differ but they have a
common notion of sequentiality, encoded with the \textbf{bind} operator
(\texttt{($\bindOp$) :: Monad m $\Rightarrow$ m a $\rightarrow$ (a $\rightarrow$
  m b) $\rightarrow$ m b}), which connects an action performed with the monad
and a continuation that requires the result of this interaction, and the ability
to embed pure values into the monad (\texttt{pure :: Monad m $\Rightarrow$ a
  $\rightarrow$ m a}).

Of particular importance is the sequentiality of actions that is enforced by
$\bindOp$. Since the left hand argument is a continuation the right hand action
must be performed before the program can advance. This structure enforces an
order to the execution of side effects whether or not these actions produce
actual results, such as writing to a table in a database.

The Monad proved to be a very successful concept in describing
sequential interactions with ambient environments and thus many
different monads had soon been developed. Monads for interacting with
the system (\texttt{IO}\footnote{Part of the \texttt{base} library}),
for interacting with databases (\RedisM{}~\footnote{Part of the
  \texttt{hedis}\cite{hedis} package for interacting with the Redis
  Key-Value-Store}) or handling web requests in a server
(\HandlerForM{}\footnote{Part of the Yesod~\cite{yesod} web
  framework}) as well as monads that defined environments embedded in
the program such as \texttt{Writer}, which collects outputs,
\ReaderM{}, which adds a static environment and
\ExceptM{}\footnote{\ExceptM{} does not actually exist. Only the
  transformer \ExceptT{} does as \ExceptM{} would be the same as
  \EitherM{}. \ExceptM{} is only used here to make the connection to
  its transformer more obvious}, which adds throwing and handling of
user defined errors.

Though each of these monads are well suited for interacting with the
various environments they describe, they are rather unwieldy, if not
impossible to use when we wish to interleave their effects. We may
desire to read some data from the \RedisM{} database, perform some
network \IOM{} afterwards and finally send some computed result as
part of being a \HandlerForM{}, all the while tracking potential
\ExceptM{} errors.

This however is not possible with a simple monad. Most of these monads
are opaque types, results from which can only be obtained by
performing complex set-up and tear-down operations, and some, like the
\HandlerForM{} monad offer no way for the user to extract the pure
data directly. This makes sense of course as the creation of this data
entails certain opaque interactions with the environment from which it
cannot be easily untangled. Though the power of the monad lies
precisely in leaving those interactions opaque it poses a challenge
when trying to achieve composability and interleaving of effects.

The next section offers an overview of the current most popular
solution for achieving composability: effect classes and monad
transformers.
