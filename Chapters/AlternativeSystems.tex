\section{Alternative Effect Systems}

\label{sec:alternative-systems}

Effects are a common topic in the field of pure functional programming
languages. Particularly the recent years have enjoyed the development of several
new effect systems with a similar aim to the freer monads. Namely providing
extensible and performant effect systems.

\subsection{Handlers in Action}

\citeauthor{hia}~\cite{hia} Developed a an alternative system, building on free
monads, similar to the one from \citeauthor{freer}~\cite{freer}. The system,
called ``Handlers in Action'' (HIA), also leverages an augmented free monad, see
figure~\ref{fig:freer-monad}, the performance of which is enhanced by using the
continuation monad \texttt{Cont}, a construct which is very similar to the
\texttt{Codensity} monad as described before and fulfils a similar role here.
Whereas freer dealt with suites of effects, \texttt{ConsoleIO} for instance is
a suite consisting of two effects, \texttt{ReadLine} and \texttt{WriteLine},
``Handlers in Action'' defines every effect individually. Every effect thus
becomes its own top-level construct. Handler functions can choose which set of
effects to handle, as well as whether additional effects, which are delegated,
should be allowed or not. Handlers that allow additional effects are called
``shallow'' and the default case for freer. Handlers do not allow additional
effects are called ``deep'' handlers. Deep handlers can provide performance
benefits due to fewer dispatches.

To implement this in a user friendly way \citeauthor{hia} leverages the code
generation tool ``Template Haskell'' to generates type classes, constraints and
data structures for effects, handlers and functions using effects.

In this approach no open union is needed. Instead each effect is directly paired
with its handler via a \texttt{Handles} type class. Since each handler function
is member of a type class it will be a static top level value, thus reducing the
memory overhead because no additional \texttt{Union} value is needed as well as
improving the speed slightly since not every handler attempts to handle every
effect, but only the appropriate handler handles its own effect. However this
depends largely on whether or not the compiler inlines the instance
dictionaries.

In the performance evaluation in \cite{freer} the HIA system performed on-par or
better than freer. In fact for the simple example of the state monad it even
outperformed MTL. As far as I can tell this is due to the fact that, unlike
freer, HIA uses more builtin structures, like the continuation monad and type
classes. These structures are more transparent to the compiler, and have been
around for a long time, thus enjoy more dedicated optimisations. Type level
sequences and generic unions, as used by freer, are opaque to a large degree and
recent additions to Haskell and hence harder to optimise by the compiler.

In addition to Haskell HIA also provides implementations for Racket and OCaml.
Due to the less powerful type systems of these two languages the implementation
of the handlers is slightly different, however the semantics of the ensuing
program are similar. They use runtime tracking of handler functions in scope,
which is a more dynamic approach. An advantage of this system is that new
operations can be defined at runtime which simplifies certain tasks. One such
example (taken from the paper~\cite{hia}) is the representation of mutable
references by a set of dynamically created \texttt{Put} and \texttt{get}
operations.

\subsection{Algebraic Effects in Idris}

Certain of the features of the effect system depend on the language used to
implement it. The Idris~\cite{idris}\cite{idris-paper} programming language is
fully dependently typed and thus more powerful than Haskell. One result of this
is that effects can be even more finely disambiguated. Whereas in freer and
extensible effects several effects of a polymorphic type can be disambiguated in
the same computation, see section~\ref{sec:freer}, in the effects system
implemented by
\citeauthor{algebraic-effects-idris}~\cite{algebraic-effects-idris} in Idris,
called ``Effects'', even effects instantiated with the exact same types can be
disambiguated by means of adding labels to a particular effect. Subsequently the handlers
are selected based on the label, not the type alone. An example of how this
proves useful is if a computation were to carry two counters which are
incremented separately. Rather than having to redefine an effect the same
handler can be used and a label to disambiguate which counter is to be
incremented.

Another key difference in Idris is that effect handlers are defined via type
classes, similar to ``Handlers in Action'', as opposed to the user-defined
handler compositions of freer. In Idris case handlers are parameterised over a
context monad in addition to the effect itself. This allows it to be more
versatile and implement different handlers for different contexts. A type class
based approach can reduce the amount of code necessary to run effects, it does
however increase the amount of boilerplate necessary to run handlers in custom
contexts, such as when mocking for testing.\footnote{The reason is that in order
  to choose different handlers a newtype has to be instantiated as base monad
  and instances provided for every effect needed. Whereas in freer predefined
  handlers can simply be composed anew.}

The effect type in Idris is also more powerful when it comes to the type of
interactions that can be described by it. Effects are parameterised over a type
of input resource and a type of output resource. These are type level constructs
that live in the environment of the effect monad. An example from the paper by
Brady~\cite{algebraic-effects-idris} is a \texttt{FileIO} effect which allows
reading and writing to files. Rather than returning some kind of file handle
when opening the file and writing to the handle the \texttt{OpenFile} action
instead records the file in the computation environment including the mode of
opening. Neither reading nor writing takes a handle as argument, but instead
retrieves the resource from the environment automatically. Similarly close file
retrieves the ambient value but also removes the resource type from the
environment. As a result a closed file can never be written to or read from,
since the type checker detects that this resource was removed from the
environment at compile time. Similarly reading and writing cannot occur on files
opened with the incorrect mode, due to the mode also being recorded in the
environment of the effect. To open several files in the same computation the
aforementioned labels are used to disambiguate the effect targets.

A downside of the Idris approach is that the effect monad is no longer a monad
conforming to the \texttt{Monad} type class, because effects may change the the
environment resources and thus the type of the computation. As an alternative
Idris supports operator overloading to leverage the convenient \texttt{do}
notation regardless. Unlike previous approaches this effect system does not use
a simple augmented free monad but a much richer data structure that encodes
several other actions in addition to the two monadic actions \texttt{pure} and
$\bindOp$. This is due to the larger feature set of ``Effects'' compared to the
systems mentioned previously.

\subsection{PureScript}

One honourable mention at this point goes to the PureScript~\cite{purescript}
language. PureScript is a pure functional language, similar to Haskell, that
compiles to JavaScript. JavaScript has a lot of different built-in effects, such
as console I/O, DOM interaction, Websockets, Browser history modification, and
even more when NodeJS is used, adding process spawning, file interactions et
cetera. To deal with such myriad of effects, while avoiding the IO-Monad
problem, as it exists in Haskell, PureScript has built in support for a an
extensible effects monad (\texttt{Eff}).

An unfortunate issue in Haskell is that all effects dealing with some kind of
external API live in the monolithic \texttt{IO} monad. A signature \texttt{IO a}
does indicate that \emph{some} kind of effect occurs, but not which. It could be
reading a file, or launch missiles, an infamous example of how arbitrary effects
are from the Haskell community. PureScript addresses this by providing the
\texttt{Eff} monad as a built-in which is parameterised over an extensible
record of effects. The interface is very similar to the freer monad from the
previous section. The difference being that all effects in PureScripts
\texttt{Eff} are native effects. Each effect type is an untouchable empty native
type with no runtime representation and it carries no data. Similarly the
tagging of the imported native functions with effects is done arbitrarily by the
user and no alternative, ``pure'' implementations directly in PureScript are
possible.

However the use \texttt{Eff} as a built-in structure as well as using an
extensible record to track the composition of effect types means that all of the
code using \texttt{Eff} can be optimised to such a degree that no overhead
remains at runtime. Both effect resolution and $\bindOp$ inlining can be done at
compile time. Furthermore having built-in extensible records result in a concise
syntax.
