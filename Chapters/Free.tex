\section{Building a Naive Extensible Monad}

\label{sec:free}

Using monad transformers involves writing significant boilerplate code defining
effect instances because every effect is implemented on a separate transformer,
which prevents sharing of the common action of lifting an effect. Lifting
through a stack can be avoided entirely by instead using a \emph{single} monad
parameterised over a variable set of effects.

\subsection{Decoupling Monad and Effect}

\label{sec:introducing-free}

\citeauthor{data-types-a-la-carte} has shown a monad which encapsulated the two
monadic actions, \texttt{pure} and $\bindOp$ as a data type with no effect
capabilities of its own~\cite{data-types-a-la-carte}. It is called a \emph{free}
monad, because it extends \emph{any} functor, which is a simpler structure, with
the monadic capabilities. Essentially the free monad represents an encoding for
the \texttt{Monad} typeclass as a recursive data type parameterised with an
effect functor. Figure~\ref{fig:free-monad} shows the implementation for reference.

\begin{figure}
  \lstinputlisting[firstline=8]{Listings/Free.hs}
  \caption{The simple free monad}
  \label{fig:free-monad}
\end{figure}

By using the free monad the definition of \texttt{pure} and $\bindOp$ can be
elided. Instead a data structure is obtained which has to be
\emph{interpreted} later, allowing the precise interpretation to vary freely
by using different interpreters. Figure~\ref{fig:console-io-example} shows a
monad for interacting with a console implemented with two different
interpreters. Crucially the computation is unaware of the different
context/interpreter, which allows free monad based systems do be easily mocked
for testing.

\begin{figure}
  \lstinputlisting[firstline=7]{Listings/ConsoleIO.hs}
   \caption{Implementing a simple console I/O interaction using
    \texttt{Free}}
  \label{fig:console-io-example}
\end{figure}

There is a caveat to using functors rather than monads for effects however.
Certain effects, such as \texttt{listen}, an effect belonging to the family of
\texttt{Writer} effects, cannot be expressed as a functor and thus cannot be
handled by a free monad based implementation.

\subsection{Composing Effects}

\label{sec:simple-effect-composition}

Functors unlike monads are composable. A simple functor providing this feature
is the \texttt{Sum} functor, here called \texttt{:+:}, see
figure~\ref{fig:sum-functor}. It uses the same technique as the \texttt{OR} type
by \citeauthor{data-types-a-la-carte}~\cite{data-types-a-la-carte} lifted to
\texttt{* $\rightarrow$ *} kinded types. By nesting \texttt{:+:} values
arbitrarily many functors can be composed. Such variable composites are called
\emph{open union}, as opposed to static composites, such as algebraic datatypes.

Similar to the transformer classes an interface is defined to interact with
specific functors inside the composition. The subtype relation class can be seen
in figure~\ref{fig:dispatch-class}. Unlike with transformer classes the subtype
relation can be defined generically without knowing the concrete type of the
functor being queried for.

\begin{figure}
  \lstinputlisting[firstline=5,lastline=9]{Listings/SimpleOpenUnion.hs}
  \caption{The sum functor \texttt{:+:}}
  \label{fig:sum-functor}
\end{figure}

\begin{figure}
  \lstinputlisting[firstline=13]{Listings/SimpleOpenUnion.hs}
  \caption{Dispatch classes for effects in a simple open union}
  \label{fig:dispatch-class}
\end{figure}

This open union is used as base functor for \texttt{Free} to form a monad with a
variable set of effects. Computations are parameterised over the types of
effects they use, via the subtype relation, without directly describing the
structure of the functor. Crucially there is only one subtype relation class
which is used for all effects \emph{without} the need for additional instances
for new effects. An extensible version of the \texttt{readLine} and
\texttt{writeLine} functions from the earlier example can be seen in figure
\ref{fig:extensible-console-effect}. Whereas the earlier version from figure
\ref{fig:console-io-example} constrained us to use the explicit type of
\texttt{Free Console}, the extensible version allows any version of
\texttt{Free}, so long as the effect functor contains the \texttt{Console}
effect. Because of the \texttt{f :<: f} instance of the dispatch class
\texttt{:+:} this also includes \texttt{Console} itself.

\begin{figure}
  \lstinputlisting[firstline=8]{Listings/ExtensibleConsole.hs}
  \caption{An extensible version of the console interations}
  \label{fig:extensible-console-effect}
\end{figure}

Similar to monad transformers computations using a subset of effects can always
be used in functions constrained to a superset of effects.

A unique feature of this extensible monad is the \texttt{interpose} combinator
which is used to intercept and possibly alter, filter or relay requests. Whereas
transformers would necessitate the provision of a separate implementation of
\texttt{interpose} for each transformer, here we are able to implement a generic
combinator usable with all effect types and agnostic to the interpreter used.

To illustrate the usefulness of this feature figure~\ref{fig:attach-timestamp}
shows a function which sanitises the input sent to the console for any monad
providing the console effect.

\begin{figure}
  \lstinputlisting[firstline=13,lastline=23]{Listings/Interpose.hs}
  \caption{The interpose combinator}
  \label{fig:interpose-combinator}
\end{figure}

\begin{figure}
  \lstinputlisting[firstline=37]{Listings/Interpose.hs}
  \caption{Two examples for using the interpose combinator}
  \label{fig:attach-timestamp}
\end{figure}

\subsection{Handling Effects}

With the extensible effects monad generic combinators can be defined to aid the
handling of effects. This is possible due to the fact that both the structure of
the monad as well as a fair bit about the request structure is known.

Most useful is the \texttt{interpret} and \texttt{reinterpret} combinator, shown
in figure~\ref{fig:interpreting}. With \texttt{reinterpret} the effect is
handled by inserting a new effect and handling the old effect in terms of this
new effect and the old effects. New effects can therefore be handles by
``remapping'' onto another effect for which a (generic) handler already exists.
However the computation will remain unaware of the added effect \texttt{g}.
\texttt{interpret} is similar but adds no new effect. Both combinators allow the
user to leverage knowledge about other effects present to relay effects or
leverage generic effects to implement specialised ones.
Figure~\ref{fig:run-console} shows an example how the console effect is
interpreted once by delegating to the \texttt{IO} monad and once by
reinterpreting it as a state effect and using the generic \texttt{runState}
interpreter.

\begin{figure}
  \lstinputlisting[firstline=10]{Listings/Interpret.hs}
  \caption{Combinators for interpreting effects}
  \label{fig:interpreting}
\end{figure}

\begin{figure}
  \lstinputlisting[firstline=37]{Listings/RunConsole.hs}
  \caption{Interpreting the extensible console effect}
  \label{fig:run-console}
\end{figure}
