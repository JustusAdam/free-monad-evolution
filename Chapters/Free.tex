\section{Freedon to the Rescue}

\label{sec:free}

In Section~\ref{sec:monads} the monad was introduced as a way to
sequence effectful computations. Then various types of monads were
found, encoding different effects.

\subsection{Decoupling Monad and Effect}

\label{sec:introducing-free}

One of the most important discoveries \textbf{CITATION} was that the
effect and its sequential nature can be separated and the latter
encoded with a generic data structure. The essence of the monad,
$\bindOp$ and \texttt{pure} is captured using the so called
\textbf{free monad} (see figure~\ref{fig:free-monad}). This generic
monad is parameterized by an effect type \texttt{f} which itself only
needs to be a \Functor{}. Functors are much simpler structures than
monads. They have no notion of sequentiality and continuations which
makes them easier to implement. A Functor only needs to allow the
application of a function to a contained value as illustrated by the
type class \Functor{} displayed in figure~\ref{fig:functor-class}.

\begin{figure}
  \begin{lstlisting}
    class Functor f where
      fmap :: (a -> b) -> f a -> f b
  \end{lstlisting}
  \caption{The \Functor{} interface}
  \label{fig:functor-class}
\end{figure}
\begin{figure}
  \begin{lstlisting}
    data Free f a
      = Impure (f (Free f a))
      | Pure a
  \end{lstlisting}
  \caption{The simple free monad}
  \label{fig:free-monad}
\end{figure}
\begin{figure}
  \begin{lstlisting}
    join :: Monad m => m (m a) -> m a
    join ma = ma >>= id

    (>>=) :: Monad m => m a -> (a -> m b) -> m b
    ma >>= cont = join $ fmap cont ma
  \end{lstlisting}
  \caption{The equivalence of $\bindOp$ and \texttt{join}}
  \label{fig:join-bind-equivalence}
\end{figure}
\begin{figure}
  \begin{lstlisting}
    instance Functor f => Monad (Free f) where
      pure = Pure
      Pure a >>= cont = cont a
      Impure fa >>= cont = Impure $ fmap (>>= cont) fa
  \end{lstlisting}
  \caption{The \Monad{} instance for \texttt{Free}}
  \label{fig:free-monad-monad-instance}
\end{figure}

The two constructors for the free monad represent the two
characteristic functions of the \Monad{} in category theory,
\texttt{pure} and \texttt{join}, the latter represented by the
\texttt{Impure} constructor. The \texttt{join :: Monad m => m (m a) ->
  m a} function allows flattening of two stacked, identical monadic
contexts. It, in conjuction with \texttt{pure} is sufficient to
characterize a monad the same way that $\bindOp$ does since either one
can be implemented in terms of the other, see
figure~\ref{fig:join-bind-equivalence}.

In case of the \texttt{Impure} constructor the continuation
(\texttt{Free f a}) is contained in a layer of the effect functor
\texttt{f} meaning that proceeding with the computations requires
execution of one instance of the effect. Only then is an
interpretation of the inner (later) computation possible, including
the effects used later in the computation. This ensures effects are
executed in the same order that they are used in, in the program.

Using this data structure removes the need define a \Monad{} instance
for the effect type \texttt{f}, as the free monad is a true \Monad{}
for \emph{any} \Functor{} \texttt{f}, see
figure~\ref{fig:free-monad-monad-instance}, this also means all the
useful combinators defined for monads are available to an effect
system defined like this. Effects are entirely delegated to the
functor \texttt{f} allowing completely independent and different kinds
of effect systems to be implemented using \texttt{Free}.

To run a computation expressed in a free monad effect system is done
by a function typically called an \emph{interpreter}. This interpreter
implements how the effects should be handled in a certain
environment. One additional advantage of using an interpreter is that
effects can be handled differently depending on the concrete
interpreter. They can even run in different environments entirely
without requiring a change in the computations code.

\begin{figure}
  \begin{lstlisting}
    data Console a
      = ReadLine (String -> a)
      | WriteLine String a

    instance Functor Console where
      fmap f (ReadLine cont) = ReadLine $ fmap f cont
      fmap f (WriteLine str a) = WriteLine str $ f a

    type ConsoleM = Free Console
      
    readLine :: ConsoleM String
    readLine = Impure (ReadLine id)

    writeLine :: String ConsoleM ()
    writeLine str = Impure (WriteLine str ())

    interpretIO :: ConsoleM a -> IO a
    interpretIO (Pure a) = pure a
    interpretIO (Impure eff) =
      case eff of
        ReadLine f -> IO.readLine >>= interpretIO . f
        WriteLine str cont = do
          putStrLn str
          interpretIO cont
        
    interpretPure :: [String] -> ConsoleM a
                  -> (a, [String])
    interpretPure lines cio = go lines [] cio
      where
        go _ outLines (Pure v) = (v, reverse outLines)
        go inLines (Impure eff) =
          case eff of
            ReadLine cont ->
              go (tail inLines) outlines
                $ cont (head inLines)
            WriteLine str cont ->
              go inLines (str:outLines) cont
  \end{lstlisting}
  \caption{Implementing a simple console I/O interaction using
    \texttt{Free}}
  \label{fig:console-io-example}
\end{figure}

An example can be seen in figure~\ref{fig:concole-io-example} where a
simple console interaction is implemented using the free monad. Here
two very different interpreters are provided. One where the
interpretation happens in the IO monad and the two requests
\texttt{ReadLine} and \texttt{WriteLine} are directly delegated to the
usual I/O functions and a second one where the requests are served
from a list of input strings and the written lines are again recorded
in a list. This power of the free monad effects to be interpreted in
different ways makes it ideal to implement mock interpreters for
testing functions as well as making code more reusable in different
scenarios. Transformers for instance do not offer the same degree of
flexibility.

\subsection{Composing Effects}

\label{sec:simple-effect-composition}

Decoupling the monad from its effect enables the composition of
effects using functor composition. Two effect systems implemented
using the free monad can be composed into an effect system
implementing both types of effects by composing the two effect
functors into a single one and using the result functor to
parameterize the free monad. When handling the effect the two functors
are decomposed again and the individual effects are dispatched to
corresponding handlers.

\begin{figure}
  \begin{lstlisting}
    data Sum f g a
      = Left (f a)
      | Right (g a)

    instance (Functor f, Functor g)
             => Functor (Sum f g) where
      fmap f (Left fa) = Left $ fmap f fa
      fmap f (Right ga) = Right $ fmap f ga
  \end{lstlisting}
  \caption{Then \texttt{Sum} functor}
  \label{fig:sum-functor}
\end{figure}

In its simplest form this is done using the 
