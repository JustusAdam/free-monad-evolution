\documentclass[10pt,a4paper,sigconf]{acmart}

\usepackage{booktabs} % For formal tables
\usepackage[utf8]{inputenc}
\usepackage{listings}
\usepackage[inline]{enumitem}
\usepackage{FiraMono}


\newcommand{\defineMonad}[2]{\newcommand{#1}{\texttt{#2}}}

\defineMonad{\ReaderM}{Reader}
\defineMonad{\WriterM}{Writer}
\defineMonad{\IOM}{IO}
\defineMonad{\StateM}{State}
\defineMonad{\ExceptM}{Except}
\defineMonad{\RedisM}{Redis}
\defineMonad{\HandlerForM}{HandlerFor}
\defineMonad{\EitherM}{Either}
\defineMonad{\IdentityM}{Identity}

\defineMonad{\ExceptT}{ExceptT}
\defineMonad{\ReaderT}{ReaderT}
\defineMonad{\StateT}{StateT}
\defineMonad{\WriterT}{WriterT}


\newcommand{\bindOp}{\gg=}
\newcommand{\bindIgnOp}{\gg}

\newcommand{\Functor}{\texttt{Functor}}
\newcommand{\Monad}{\texttt{Monad}}


\lstset{%
  language=Haskell,
  basicstyle=\small\ttfamily,
  literate=
  {->}{{$\rightarrow$}}1 {=>}{{$\Rightarrow$}}1 {>>=}{{$\bindOp$}}2 {>>}{{$\bindIgnOp$}}1
  {<-}{{$\leftarrow$}}1
}

% Copyright
%\setcopyright{none}
%\setcopyright{acmcopyright}
%\setcopyright{acmlicensed}
\setcopyright{rightsretained}
%\setcopyright{usgov}
%\setcopyright{usgovmixed}
%\setcopyright{cagov}
%\setcopyright{cagovmixed}

\begin{document}
\title{Why Data is the Better Monad}
\subtitle{Using Freedom to Great Effect}

\author{Justus Adam}
\affiliation{}
\email{justus.adam@tu-dresden.de}


\begin{abstract}
  Monads are an abstraction used to express side effects in pure languages such
  as Haskell and Idris. Due to their structure they do not compose well, and
  state of the art solutions to this problem involve lots of boilerplate code.

  ``Extensible effect'' systems utilise a single Monad, parameterised by a set
  of monadic effects. Effects are expressed as data structures, composed via an
  open union, allowing dynamic addition, subtraction and rewriting of effects.
  Unlike previous approaches, computations can be implemented in terms of a
  \emph{single} generic member constraint on the effect set avoiding per-effect
  boilerplate classes and instances.

  Furthermore new effects are easier to define and effect interpreters can be
  selected dynamically.
\end{abstract}

\maketitle

\section{Introduction}

\label{sec:introduction}

% \section{Monads}

\label{sec:transformers}

Monads are an abstraction used to implement effects on an ambient environment in
languages such as Haskell and Idris~\cite{wadler-monads}. Inherently monads must
only satisfy and interface of two function, \texttt{pure :: Monad m
  $\Rightarrow$ a $\rightarrow$ m a} to embed any value in the monad, and
\texttt{($\bindOp$) :: Monad m $\Rightarrow$ m a $\rightarrow$ (a $\rightarrow$
  m b) $\rightarrow$ m b}, also called \emph{bind}, to sequence two monadic
computations. While this captures the sequential nature of the monad, it does
not provide any way to extract a value, or combine one monad with another,
because the chained operation must live in the same monad as the value it is
chained to.


Purity, often also called referential transparency, is the property of
a function to always produce the same result when called with the same
arguments. It is a desirable property as it makes it considerably
easier to reason about the behaviour of a program, particularly with
respects to refactoring. But it also allows optimisations, such as
common subexpression elimination and memoization.

However, many tasks that we wish programs to perform cannot be
expressed in terms of pure functions as they entail an interaction
with the world outside of the program, such as accessing a database or
querying the file system. Even inside of the program it is often
useful to define some ambient environment in which certain tasks are
performed, from this object oriented programming was conceived where
each function, or method, carries around the implicit environment of
an object.

To perform the aforementioned tasks many programming languages opt to sacrifice
purity and allow arbitrary side effects in any function. The designers of the
Haskell programming language however found a way to enable side effects in
programs without sacrificing purity. \textbf{Monads}, in the Haskell sense, are
a class of types that describe some sort of environment, which can be interacted
with. The concrete interactions a particular monad allows differ but they have a
common notion of sequentiality, encoded with the \textbf{bind} operator
(\texttt{($\bindOp$) :: Monad m $\Rightarrow$ m a $\rightarrow$ (a $\rightarrow$
  m b) $\rightarrow$ m b}), which connects an action performed with the monad
and a continuation that requires the result of this interaction, and the ability
to embed pure values into the monad (\texttt{pure :: Monad m $\Rightarrow$ a
  $\rightarrow$ m a}).

Of particular importance is the sequentiality of actions that is enforced by
$\bindOp$. Since the left hand argument is a continuation the right hand action
must be performed before the program can advance. This structure enforces an
order to the execution of side effects whether or not these actions produce
actual results, such as writing to a table in a database.

The Monad proved to be a very successful concept in describing
sequential interactions with ambient environments and thus many
different monads had soon been developed. Monads for interacting with
the system (\texttt{IO}\footnote{Part of the \texttt{base} library}),
for interacting with databases (\RedisM{}~\footnote{Part of the
  \texttt{hedis}\cite{hedis} package for interacting with the Redis
  Key-Value-Store}) or handling web requests in a server
(\HandlerForM{}\footnote{Part of the Yesod~\cite{yesod} web
  framework}) as well as monads that defined environments embedded in
the program such as \texttt{Writer}, which collects outputs,
\ReaderM{}, which adds a static environment and
\ExceptM{}\footnote{\ExceptM{} does not actually exist. Only the
  transformer \ExceptT{} does as \ExceptM{} would be the same as
  \EitherM{}. \ExceptM{} is only used here to make the connection to
  its transformer more obvious}, which adds throwing and handling of
user defined errors.

Though each of these monads are well suited for interacting with the
various environments they describe, they are rather unwieldy, if not
impossible to use when we wish to interleave their effects. We may
desire to read some data from the \RedisM{} database, perform some
network \IOM{} afterwards and finally send some computed result as
part of being a \HandlerForM{}, all the while tracking potential
\ExceptM{} errors.

This however is not possible with a simple monad. Most of these monads
are opaque types, results from which can only be obtained by
performing complex set-up and tear-down operations, and some, like the
\HandlerForM{} monad offer no way for the user to extract the pure
data directly. This makes sense of course as the creation of this data
entails certain opaque interactions with the environment from which it
cannot be easily untangled. Though the power of the monad lies
precisely in leaving those interactions opaque it poses a challenge
when trying to achieve composability and interleaving of effects.

The next section offers an overview of the current most popular
solution for achieving composability: effect classes and monad
transformers.

\section{Transformers and Classes}

\label{sec:transformers}

The need for ambient effects is different for each program. Whereas effect
monads like \IOM{} and \HandlerForM{} are absolutely required, as they provide
the only facility to interact with certain resources, additional ``convenience''
monads such as \ReaderM{}, \WriterM{}, \StateM{}\footnote{Similar to the
  \ReaderM{} monad but the environment is not static and can be altered during
  the computation} and \ExceptM{} are ``optional'' and provide easier interfaces
for describing certain types of computation.

For each domain there is a different set of effect combinations particularly
suited to describe it. When terminal monads are used, like those mentioned in
the last paragraph, an entirely new monad would have to be implemented for every
combination of effects we desire. Alternatively one large monad, which includes
\textbf{all} effects could be used, but then each effect would necessarily have
to be handled when the monad is run, regardless of whether the computation
actually uses it. And how would the domain specific effects like \IOM{} and
\HandlerForM{} be added to this monolithic monstrosity?

Clearly neither of these approaches is well suited to solve the need for
interleaved effects. A system is needed whereby larger, more complicated, monads
can be composed of smaller, elemental ones. Ideally this could be done
dynamically, as some sections of the code may require additional effects to be
added, like adding a \WriterM{} to accumulate the results of some embedded
computation.

The most popular solution currently is the one implemented by the
\texttt{mtl}\cite{mtl} and \texttt{transformers}\cite{transformers} library and
it revolves around using stacks of so called \textbf{monad transformers} and
\textbf{monad classes}. The approach was inspired by a paper from Mark P.
Jones~\cite{transformer-inspiration}. The idea is that, rather than defining
terminal monads, such as \ReaderM{}, the defined monad is parameterized by an
additional \emph{inner monad} to which other effects can be delegated. The
resulting structure is called a \emph{transformer} as it transforms a given
monad by adding an additional type of effect. Transformers are typically named
after their effect with a capital ``T'' prepended. Thus the transformer for the
\ReaderM{} monad would be \ReaderT{}.

This process of wrapping a monad with a transformer can be done arbitrarily
often as the result of wrapping a monad with a transformer again produces a
monad which can become the inner monad to another transformer. The resulting
structure is often also referred to as a transformer \emph{stack} that is
terminated at the last level by some terminal monad which cannot be used as a
transformer, such as \IOM{} or \HandlerForM{}.

When the computation is executed the layers of transformers are removed
individually, the outermost layer always first until a computation in the base
monad remains. Transformer stacks can even be used to describe pure computations
by choosing \IdentityM{} as the inner monad, which resolves to a pure value.
Additionally, during a computation, new (outer) layers can be dynamically added
and removed from the monad. As such for instance a subcomputation can be run
with an additional \WriterM{} effect, that is then separately unwrapped to
obtain the written results.

The characteristic action of a transformer, as captured by the
\texttt{MonadTrans} class, is \texttt{lift :: (MonadTrans n, Monad m)
  $\Rightarrow$ m a $\rightarrow$ n m a} which can be seen as embedding an
action of the inner monad in the outer monad, or, alternatively, as delegating
(lifting) an effect to be handled by the inner monad. Unfortunately this lifting
operation poses a problem when the transformer stack get larger. Performing an
effect which is far up the stack will require several lifting operations to
reach the transformer that can actually satisfy it.

To deal with this issue another concept is used in the form of \textbf{monad
  classes}. Rather than defining effects on a transformer directly a type class
is defined with the interface for the effect in question, see for instance the
effect as provided by the \StateM{} monad in
figure~\ref{fig:monad-class-example}. Transformers responsible for handling the
effect are then made instances of this class and implement it accordingly.
Additionally all other transformers are also made instances of this class but
use \texttt{lift} to delegate the effect up the stack. Thus when such an effect
is used it automatically propagates up the stack until it reaches the
transformer capable of handling it (should one exist\footnote{Should no capable
  handler exist the type checker will report a class instance resolution error.}).

\begin{figure}
  \begin{lstlisting}
    class MonadState s m | m -> s where
      put :: s -> m ()
      get :: m s
  \end{lstlisting}
  \caption{The monad class for the \StateM{} effect}
  \label{fig:monad-class-example}
\end{figure}

An additional advantage of monad classes is that effectful computations can be
defined solely in terms of the set of effects it uses, rather than the concrete
monad it operates in. As a result they can be reused in entirely different monad
stacks without modification of the code so long as it satisfies the same effect
interface. An example of this can be seen in figure~\ref{fig:monad-class-poly}.
Whereas the second function, \texttt{writeState1}, can only be used in the
explicit \StateT{}, \WriterT{} stack, the first function, \texttt{writeState},
could also be used in a stack where the two transformers are reversed, or if it
were wrapped by additional transformers.

\begin{figure}
  \begin{lstlisting}
    writeState :: ( MonadState s m
                  , MonadWriter s m
                  , Monoid s )
               => m ()
    writeState = get >>= tell >> put mempty

    writeState1 :: ( Monad m
                   , Monoid s )
                => StateT s (WriterT s m) ()
    writeState = get >>= tell >> put mempty
  \end{lstlisting}
  \caption{Polymorphic and explicit effect signatures}
  \label{fig:monad-class-poly}
\end{figure}

While this provides a clean interface for interacting with the various effects
in our stack, using the class-based effect functions like \texttt{get}, it also
poses some issues. In the interface definition for the \StateM{} effect, for
instance, the type of state, which can be handled, is fixed to the monad, as
expressed by the \texttt{m $\rightarrow$ s} functional constraint. This means a
stack can only ever handle the state effect for \emph{one single type of state
  $s$}. Adding additional \StateT{} transformer will shadow any \StateM{} effect
of its inner monad and render it unreachable using the \texttt{MonadState}
class~\footnote{Theoretically it is still reachable using a combination of
  explicit \texttt{lift} and the effect function but using \texttt{lift}
  reintroduces the issues from before that prompted the addition of monad
  classes in the first place.}. This becomes particularly troublesome when
libraries are used that implement certain interactions in terms of effects on a
specific type, like \ReaderM{} on a specific type \texttt{t} but the monad it is
supposed to run in is already a \ReaderM{} for some \emph{other} type. In these
cases it becomes rather tedious to wrap and unwrap every library interaction
with another \ReaderT{} to shadow ones own \ReaderM{} effect. This problem can
be eased using lenses but requires that the library be defined both in terms of
lenses as well as monad classes.

Monad class instances themselves also pose issues. In general it is beneficial
to define custom effects in terms of new monad classes and transformers. This
makes these effects extensible, composable and reusable, properties which are
desired, particularly in large projects which are frequently subjected to
refactoring and expansion.

However defining you own transformers is tedious. Not only does one have to
implement the transformer itself but additionally provide a monad class instance
for every \emph{type of effect} one could possibly combine with the new
transformer. Additionally a new class for the custom effect as well es instances
of this class for every \emph{existing transformer} is necessary to achieve the
maximum amount of composability. In case of GHC some of these can be
automatically derived by the compiler, but only when \texttt{newtype} wrapping
is used. When the granularity of the custom effects is decreased, that is they
are broken into smaller and smaller pieces to achieve modularity, more and more
of such definitions and instances are required, bloating the code.\footnote{One
  particularly good example of this I believe is the \texttt{monad-logger}
  library, paticularly this module
  https://github.com/snoyberg/monad-logger/blob/master/Control/Monad/Logger.hs
  where they even resort to the use of preprocessor macros to reduce the code
  duplication.}

A truly extensible system should not require changes to existing effects in
order to incorporate new ones. Ideally we would only have to define our effect
interface and a means to perform them and the system would implicitly know how
to combine it all other effects.

The next section introduces a first such system by first removing the
boilerplate of defining monads in the first place and then moves on to extend
this into a composable system of effects.

\section{Freedom to the Rescue}

\label{sec:free}

In Section~\ref{sec:monads} the monad was introduced as a way to
sequence effectful computations. Then various types of monads were
found, encoding different effects.

\subsection{Decoupling Monad and Effect}

\label{sec:introducing-free}

One of the most important discoveries \textbf{CITATION} was that the
effect and its sequential nature can be separated and the latter
encoded with a generic data structure. The essence of the monad,
$\bindOp$ and \texttt{pure} is captured using the so called
\textbf{free monad} (see figure~\ref{fig:free-monad}). This generic
monad is parameterized by an effect type \texttt{f} which itself only
needs to be a \Functor{}. Functors are much simpler structures than
monads. They have no notion of sequentiality and continuations which
makes them easier to implement. A Functor only needs to allow the
application of a function to a contained value as illustrated by the
type class \Functor{} displayed in figure~\ref{fig:functor-class}.

\begin{figure}
  \begin{lstlisting}
    class Functor f where
      fmap :: (a -> b) -> f a -> f b
  \end{lstlisting}
  \caption{The \Functor{} interface}
  \label{fig:functor-class}
\end{figure}
\begin{figure}
  \lstinputlisting[firstline=8]{Listings/Free.hs}
  \caption{The simple free monad}
  \label{fig:free-monad}
\end{figure}
\begin{figure}
  \begin{lstlisting}
    join :: Monad m => m (m a) -> m a
    join ma = ma >>= id

    (>>=) :: Monad m => m a -> (a -> m b) -> m b
    ma >>= cont = join $ fmap cont ma
  \end{lstlisting}
  \caption{The equivalence of $\bindOp$ and \texttt{join}}
  \label{fig:join-bind-equivalence}
\end{figure}

The two constructors for the free monad represent the two
characteristic functions of the \Monad{} in category theory,
\texttt{pure} and \texttt{join}, the latter represented by the
\texttt{Impure} constructor. The \texttt{join :: Monad m => m (m a) ->
  m a} function allows flattening of two stacked, identical monadic
contexts. It, in conjuction with \texttt{pure} is sufficient to
characterize a monad the same way that $\bindOp$ does since either one
can be implemented in terms of the other, see
figure~\ref{fig:join-bind-equivalence}.

In case of the \texttt{Impure} constructor the continuation
(\texttt{Free f a}) is contained in a layer of the effect functor
\texttt{f} meaning that proceeding with the computations requires
execution of one instance of the effect. Only then is an
interpretation of the inner (later) computation possible, including
the effects used later in the computation. This ensures effects are
executed in the same order that they are used in, in the program.

Using this data structure removes the need define a \Monad{} instance
for the effect type \texttt{f}, as the free monad is a true \Monad{}
for \emph{any} \Functor{} \texttt{f}, see
figure~\ref{fig:free-monad}, this also means all the
useful combinators defined for monads are available to an effect
system defined like this. Effects are entirely delegated to the
functor \texttt{f} allowing completely independent and different kinds
of effect systems to be implemented using \texttt{Free}.

To run a computation expressed in a free monad effect system is done
by a function typically called an \emph{interpreter}. This interpreter
implements how the effects should be handled in a certain
environment. One additional advantage of using an interpreter is that
effects can be handled differently depending on the concrete
interpreter. They can even run in different environments entirely
without requiring a change in the computations code.

\begin{figure}
  \lstinputlisting[firstline=7]{Listings/ConsoleIO.hs}
   \caption{Implementing a simple console I/O interaction using
    \texttt{Free}}
  \label{fig:console-io-example}
\end{figure}

An example can be seen in figure~\ref{fig:console-io-example} where a
simple console interaction is implemented using the free monad. Here
two very different interpreters are provided. One where the
interpretation happens in the IO monad and the two requests
\texttt{ReadLine} and \texttt{WriteLine} are directly delegated to the
usual I/O functions and a second one where the requests are served
from a list of input strings and the written lines are again recorded
in a list. This power of the free monad effects to be interpreted in
different ways makes it ideal to implement mock interpreters for
testing functions as well as making code more reusable in different
scenarios. Transformers for instance do not offer the same degree of
flexibility.

\subsection{Composing Effects}

\label{sec:simple-effect-composition}

Decoupling the monad from its effect enables the composition of
effects using functor composition. Two effect systems implemented
using the free monad can be composed into an effect system
implementing both types of effects by composing the two effect
functors into a single one and using the result functor to
parameterize the free monad. When handling the effect the two functors
are decomposed again and the individual effects are dispatched to
corresponding handlers.

\begin{figure}
  \lstinputlisting[firstline=5,lastline=9]{Listings/SimpleOpenUnion.hs}
  \caption{Then sum functor \texttt{:+:}}
  \label{fig:sum-functor}
\end{figure}

The simplest way to combine effects is the sum functor (\texttt{:+:}) as seen in
figure \ref{fig:sum-functor}. If combines two functors \texttt{f} and
\texttt{g}, dispatching requests accordingly. We can use this simple union of
two types to express functors that combine arbitrary types by nesting
\texttt{:+:} and add projection (\texttt{prj}) and injection (\texttt{inj}) via
a subtype relation class \texttt{:<:}, see figure \ref{fig:dispatch-class}. This
way we form a so called \emph{Open Union} of functors. The sum functor and the
open union is a technique developed by Wouter
Swierstra~\cite{data-types-a-la-carte} and based on the \texttt{OR} type from
Liang, Hudak and Jones~\cite{monad-transformers}, lifted to the higher kinded
functors.

\begin{figure}
  \lstinputlisting[firstline=13]{Listings/SimpleOpenUnion.hs}
  \caption{Dispatch classes for effects in a simple open union}
  \label{fig:dispatch-class}
\end{figure}

Using an open union like this as the base functor for the free monad allows us
to parameterize functions over the types of effects they use without directly
describing the structure of the functor and without the need for additional type
classes and class instances for every new effect. An extensible version of the
\texttt{readLine} and \texttt{writeLine} functions from the earlier example can
be seen in figure \ref{fig:extensible-console-effect}. Whereas the earlier
version from figure \ref{fig:console-io-example} constrained us to use the
explicit type of \texttt{Free Console}, the extensible version allows any
version of \texttt{Free}, so long as the effect functor contains the
\texttt{Console} effect. Because of the \texttt{f :<: f} instance of the
dispatch class \texttt{:+:} this also includes \texttt{Console} itself.

\begin{figure}
  \lstinputlisting[firstline=8]{Listings/ExtensibleConsole.hs}
  \caption{An extensible version of the console interations}
  \label{fig:extensible-console-effect}
\end{figure}

As the functions are now independent of the concrete type of underlying functor
we can combine effects freely. In figure \ref{fig:time-effect} for instance we
can see another effect which retrieves the current time. A function that wishes
to use both effects can now do so, again without having to specify any
particular structure of the functor but simply gaining a constraint, see figure
\ref{fig:combined-console-time-effect}.

\begin{figure}
  \lstinputlisting[firstline=9]{Listings/TimeM.hs}
  \caption{An effect for retrieving the current time}
  \label{fig:time-effect}
\end{figure}

\begin{figure}
  \lstinputlisting{Listings/CombinedTimeConsole.hs}
  \caption{A function using both \texttt{Console} and \texttt{Time}}
  \label{fig:combined-console-time-effect}
\end{figure}

In addition the combination of effects we can also implement an
\texttt{interpose} combinator which can be used to alter effects, see
figure \ref{fig:interpose-combinator}. This can be used for instance to attach a
timestamp to each message sent to the console by use of the \texttt{Time} effect
or implement input filtering, see figure \ref{fig:attach-timestamp}.

\begin{figure}
  \lstinputlisting[firstline=13,lastline=22]{Listings/Interpose.hs}
  \caption{The interpose combinator}
  \label{fig:interpose-combinator}
\end{figure}

\begin{figure}
  \lstinputlisting[firstline=24]{Listings/Interpose.hs}
  \caption{Two examples for using the interpose combinator}
  \label{fig:attach-timestamp}
\end{figure}

\subsection{Handling Effects}

While it is not necessary to know about the concrete structure of the effect
functor to use its effects it is necessary to handle them. The nesting of sum
functors forms a linked list and effects have to interpreted starting from the
head of the list. Various combinators are available to handle these effects. In
figure~\ref{fig:interpreting} the \texttt{interpret} combinator is shown which
interprets an effect in terms of other effects present in the list. Similarly it
shows the signatures of other combinators. Some effects, like the
\texttt{Reader} effect shown in figure \ref{fig:reader-effect} can be
interpreted directly. Others may need to be delegated. The idea is that we
either handle effects or delegate until we have unrolled the whole effect list
and are left with some base monad (often \texttt{IO}) which we run directly
using the \texttt{runM} or with no effect at all and we can simply return the
pure value with \texttt{run}, both can be seen in figure \ref{fig:interpreting}.

\begin{figure}
  \lstinputlisting[firstline=10]{Listings/Interpret.hs}
  \caption{Combinators for interpreting effects}
  \label{fig:interpreting}
\end{figure}

It may not seem immediately obvious, but this retains the same flexibility we
saw earlier in the \texttt{Free} monad whereby effects could be interpreted
differently based on context. In this case it is even easier to do this. For
instance in the example of handling the console effect our computation will have
a type of \texttt{(Console :<: sum, Functor sum) => Free sum a}. If this is to
be interpreted in the \texttt{IO} monad \texttt{sum} can be instantiated to be
\texttt{Console :+: IO} and interpreted with the \texttt{interpret} function.
All console effects are delegated using the \texttt{liftIO} helper and the two
handlers \texttt{putStrLn} and \texttt{getLine} which were also used in the
earlier example in figure~\ref{fig:console-io-example}. A computation of type
\texttt{Free IO a} remains, which we can be run using the \texttt{runM}
function. If instead the interpretation should happen in a pure context, similar
to the example before, \texttt{sum} can instead be instantiated to
\texttt{Console :+: Identity}. Using the \texttt{reinterpret} function, the
\texttt{Console} effect is turned into a \texttt{State ([String], [String])}
effect. \texttt{State} is a standard effect and its implementation omitted here.
It can be run using the \texttt{runState :: s -> Free (State s :+: sum) a ->
  Free sum (a, s)} handler. After interpreting \texttt{State} only \texttt{Free
  Identity (a, s)} is left, which fits the type signature for \texttt{run}.
Using \texttt{run} the final type will be \texttt{(a, ([String], [String]))},
which is the result of our computation, as well as the remaining, unread lines
of input and the lines written to output. Code for this example can be viewed in
figure~\ref{fig:run-console}. In short we retain the flexibility of
context-dependent interpretation because our computation is unaware of the
underlying effect layers and thus we are free to instantiate them as necessary.

\begin{figure}
  \lstinputlisting[firstline=42]{Listings/RunConsole.hs}
  \caption{Interpreting the extensible console effect}
  \label{fig:run-console}
\end{figure}

\subsection{Summary}

Using a free monad we can decouple the computation using an effect from the way
that effect is handled. Effects themselves need only concern themselves with
encoding the effect in a functor and the free monad provides the sequentiality
``for free''. By using an open union the functions using the effect can be
implemented unaware of the concrete structure of the underlying effect functor.
This means that new effects can be added in a modular fashion without the need
to change the functions using them.

Interpreting the various effects is highly flexible and can be changed depending
on the context of the computation, allowing easy mocking for instance.
Furthermore using the open union allows the interpretation of effects in terms
of other effects that can be handled easier or are generic effect types.

What was set out to do has been achieved, a highly modular, easy to extend and
flexibly interpretable system for defining effectful computations.

However an issue remains with regards to efficiency. Particularly the
implementation of an open union using a linked list of functors means that some
part of the list has to be traversed on every \texttt{fmap}, which is used
ubiquitously in the implementation. This also means that the performance
degrades as the number of different effect types grows. Smaller effect
granularity, which leads to greater reusability and makes it easier to reason
about effects, is punished by degrading performance, an undesirable effect.
Furthermore the $\bindOp{}$ implementation also recurses into the continuation on
every bind, leading to bad asymptotic performance. The next section will explore
an alternative way to implement the free monad and open union to improve the
performance.

\section{Improving the Open Union}

\label{sec:better-open-union}

For extensible effect systems to become viable it is important that its
performance is comparable with the currently employed solutions.
Monad transformers have been a part of the Haskell ecosystem for a long time and
according to \citeauthor{freer} the \texttt{State} monad in particular benefits
from dedicated optimisation passes in the Haskell compiler GHC.

The free monad in conjunction with the open union from the previous section is
unable to compete with the \texttt{mtl}. Encoding the open union of types with a
chain of sums is inefficient. Its structure is similar to a linked list and
hence it has similar computational complexity.

To construct a less space consuming open union, an encoding similar to the
encoding of closed unions is used. Efficient closed unions pair the payload
with a tag value. At runtime code dispatches based on the value of the tag.

GHC provides a built-in mechanism for generating runtime representations of
types called (\texttt{TypeRep}). \texttt{TypeRep}s support fast comparison via a
compiler generated unique 128 bit MD5 hash. Figure~\ref{fig:union-type} shows
the implementation of the reflection based union.

\begin{figure}
  \lstinputlisting{Listings/EfficientUnion.hs}
  \caption{A single value open union based on reflection}\footnote{Slightly
      simplified and with the \texttt{Member} class renamed for consistency.}
  \label{fig:union-type}
\end{figure}

While this allows for the implementation of a single-struct open union, as
opposed to the linked list from before, resolving the type still involves
several steps.
\begin{enumerate*}
\item Pointer dereference to get to the \texttt{TypeRep} structure.
\item tag comparison for the two type reps since \texttt{TypeRep} is itself an
  algebraic data type and
\item comparing the two 128 bit values.
\end{enumerate*}
Considering the generally small size the union will have a 128 bit fingerprint
is unnecessarily large.

The followup paper by \citeauthor{freer}~\cite{freer} proposes an alternative
source for a tag value, the index in the type level list. Since each union
\texttt{Union r a} carries with it the type level list \texttt{r} each functor
type \texttt{t} to which we can coerce a value must be present in this list and
this have an index that is known at compile time. This index is much smaller
than the fingerprint. The original implementation in~\cite{freer} used an
\texttt{Int} value, which is at least 30 bit, later implementations use
\texttt{Word} which is the same size, but unsigned. Unlike the \texttt{TypeRep}
from before this value can be directly inlined into the \texttt{Union}
constructor, alleviating the need for both the pointer dereference as well as
the tag comparison.

The index based approach also removes the need to derive a \texttt{Typeable}
instance for the effect type.

\section{Asymptotic performance of $\bindOp$}



As mentioned before the asymptotic performance of $\bindOp$ for the
\texttt{Free} monad is not good. As can be seen in figure~\ref{fig:free-monad}
the implementation for $\bindOp$ pushes the $\bindOp$ down the tree using
\texttt{fmap} until it reaches a \texttt{Pure} value. This means that for every
use of $\bindOp$ the entire tree must be traversed once.

\subsection{Better asymptotic performance using continuation passing}

\ref{sec:performance-with-codensity}

This problem is akin to appending to singely linked lists where the entire list
has to be traversed to find the final pointer and append. A solution to this
problem was independently developed in 2008 by Janis
Voigtländer~\cite{asymptotic-performance-improvement} in an attempt to make free
monads more efficient. The idea is rather simple and again related to the
problem of list appends. For list appends the solution is the so called
\emph{difference list}~\cite{difference-list} which really is not a list but a
function of type \texttt{[a] -> [a]}. This function will, when called with a
list, append it to its end and return the resulting list. Normal list
operations, such as appending and prepending are realised via function
composition. This is often also described as building the list while leaving a
``hole'' at its end to be plugged with a terminating value once the construction
is complete. One this value is provided, which is usually just an empty list, it
can be built in a single step. In imperative programming this is known as the
\emph{builder pattern}. A hole is left to the end of the list and all altering
operations are realised with function composition.

Asymptotic performance improvement of the free monad employs a similar idea. It
uses a generic structure from category theory, which is now known as the
\emph{Codensity monad}. The original paper by
Voigtländer~\cite{asymptotic-performance-improvement} called it \texttt{C}. The
codensity monad, shown in figure~\ref{fig:codensity-monad}, is also a function
which, given a continuation, builds some monad \texttt{m}. The \emph{extensible
  effects} library, developed by Kiselyov et al~\cite{extensible-effects} uses
this codensity monad, specialised to a variant of \texttt{Free}. The figure also
shows the new implementation for $\bindOp$ which does not push down the tree
anymore, but builds two lambda functions, which are passed as continuations, a
much cheaper operation in Haskell.

\begin{figure}
  \lstinputlisting[firstline=9]{Listings/Codensity.hs}
  \caption{The codensity monad}
  \label{fig:codensity-monad}
\end{figure}
\label{sec:bind-performance}

\subsection{Improving interpreter performance using type aligned sequences}

\label{sec:type-aligned-sequence}

The builder pattern, or programming with holes and continuations can drastically
improve the construction of a value. However this only work when the structure
is built in its entirety and then inspected in a second step. If the structure
needs to be read \emph{during} its construction this approach cannot be used.
Because a function is an opaque object, the intermediate values of a list, or in
this case a forming monad chain, cannot be read, necessitating an evaluation of
the structure before reading. Subsequent constructions can be sped up again, but
an intermediate structure was allocated to facilitate the reads, diminishing the
achieved gains.

For the effect system, implemented by our free monads, in particular, this
situation occurs in the interpreter. In order an effect we first must inspect
the computation. When the effect has been handled the continuation has to be
pushed through the monad, necessitating a full traversal of the (potentially
growing) effect chain. The crucial point where this traversal is performed can
be seen in figure~\ref{fig:ee-handle-relay} where the continuation is not simply
appended, as was the case in figure~\ref{fig:codensity-monad}, but traverses the
entire stack, similar to figure~\ref{fig:free-monad}.

The principal problem her is that new continuations, in the form on lambda
functions are created and partial applications are pushed into the functor. This
is a typical case where allocation occurs in Haskell. While these are eventually
reduced to smaller and simpler expressions, the intermediate allocations never
the less occur which increases both the time needed and the space. As the
evaluation from the extensible effects paper~\cite{freer} shows this regression
in time and space only occurs when a \emph{used} effect is threaded through the
computation, meaning the second clause in figure~\ref{fig:ee-handle-relay} is
used. In the case where the reader effect is \emph{under} the state, this clause
is triggered much less often, explaining the linear performance. Such a
performance degradation caused by the order of effects is highly undesirable. It
would be difficult to a user of this system to utilise it correctly in the
presence of such subtle differences influencing the performance.

A recent paper by van der Ploeg and Kielyov~\cite{ftc-queue} designed a solution
to this problem in the form of so called \emph{type aligned sequences}. The
approach is applicable much more broadly than only free monads. Van der Ploeg
and Kiselyov represent chains of computations with data structures via the use
of Generalized Algebraic Data Types or GADTs. Whereas usually chains of
computations are simply composed, instead they are stored in a data structure
that captures how input and return types relates, preserving the type safety. A
small example of one such structure can be seen in
figure~\ref{fig:type-aligned-list}. This affords the possibility to inspect and
alter individual links of the chain. Type safety is preserved via the GADT which
does not allow the storage of an incompatible chain segments. For the free monad
in particular a type aligned double-ended queue is used, represented internally
as a tree. Now the handler computations inspect only the head of the queue and
push new types of effects back onto the front without altering the tail.

\begin{figure}
  \lstinputlisting{Listings/TypeAlignedList.hs}
  \caption{A type aligned list of functions}
  \label{fig:type-aligned-list}
\end{figure}

With the two improvements from these last two sections the computational
complexity of extensible effects improves dramatically. Whereas the naive
implementation from section~\ref{sec:free} worse than quadratic in both time and
space, the improved version performs linearly with respect to both. This is not
only an improvement over the MTL library. According to the performance
evaluation in \cite{freer} the MTL library is \emph{faster} for a single state
effect, due to GHC specific optimisations deliberately targeted at the MTL
\texttt{State} monad. However MTL does not scale well with more effects. It
scales quadratically in both time and space, regardless of the ordering of
effects.

\section{A freer monad}

\label{sec:freer}

The ``freer'' effect system brings another advantage over the older ``extensible
effects''. As mentioned in section~\ref{sec:free} the definition of an effect
system requires an effect \texttt{Functor}. This is not strictly true. It is
convenient for the definition of a simple effect system, but can be elided. One
way to do so is using yet another kan extension, a relative of to the
\emph{codensity monad} seen in section~\ref{sec:performance-with-codensity}.
This structure is called the \emph{coyoneda functor} and it is able to lift an
arbitrary \texttt{* $\rightarrow$ *} kinded type to a functor. Simply put it
accumulates alterations applied with \texttt{fmap} in the first field. The
contained structure can be extracted, if a function is provided to apply the
accumulated alterations.

\begin{figure}
  \lstinputlisting[firstline=3]{Listings/Coyoneda.hs}
  \caption{The coyoneda functor}
  \label{fig:coyoneda}
\end{figure}

\begin{figure}
  \lstinputlisting[firstline=3]{Listings/Freer.hs}
  \caption{The augmented free monad with decoupled effect}
  \label{fig:freer-monad}
\end{figure}

While this structure can remove the functor constraint it still couples the
continuation with the effect. Kiselyov and Ishii~\cite{freer} take it one step
further and completely uncouple the two. To achieve this the free monad is
augmented such that the \texttt{Impure} case contains \emph{two} fields, one for
the effect used and one for its continuation as can be seen in
figure~\ref{fig:freer-monad}. In fact only having this decoupled monad is what
enables the use of the type aligned sequence in
section~\ref{sec:type-aligned-sequence}. Since the continuation is no longer
hidden in the effect functor and has a the general structure of \texttt{a
  $\rightarrow$ Eff eff b} it no longer needs to be an actual function, but can
be replaced by something which can be used \emph{like a function}, in case of
the freer monad a sequence of continuations.

Uncoupling effect and continuation results in a much more expressive interface
for effects. Whereas before each effect contained some additional fields for
continuations, now their mere signatures concisely express \emph{what} the
effect actually does. Freer effects use GADTs to express which forms an effect
may assume. Thus an effect is associated with some potential inputs, which are
later passed to the effect handler, via the fields of the effect value, as well
as with the type of the data returned by the effect to the computation via the
type variable parameterising the effect type. For instance in
figure~\ref{fig:console-io-freer} is the console effect from
section~\ref{sec:free} and figure~\ref{fig:console-io-example}. The two
constructors of the effect now describe, in types, what to expect, similar to a
type signature in traditional effectful computations. \texttt{ReadLine} takes no
input and returns a string, indicating that this effect will fetch some
\texttt{String} value \emph{from} the environment, whereas the
\texttt{WriteLine} effect receives a \texttt{String} as input and returns the
unit value, meaning no output is produced, which indicates that a
\texttt{String} value is delivered \emph{to} the environment.

\begin{figure}
  \lstinputlisting[firstline=4]{Listings/ConsoleIOFreer.hs}
  \caption{Freer version of the console IO effect}
  \label{fig:console-io-freer}
\end{figure}

Lastly the handler for an effect no longer has to care about the continuation.
Effect handlers are, in their simplest form, functions of type \texttt{f a $\rightarrow$
  Eff effs a}, where \texttt{f} is the effect type, for instance
\texttt{ConsoleIO}, and \texttt{effs} does not contain \texttt{f} anymore. An
example, how simple such a handler may look, can also be seen in
figure~\ref{fig:console-io-freer}.

With the \texttt{Typeable} requirement gone (see
section~\ref{sec:better-open-union}) as well as the \texttt{Functor} constraint,
expressive effect signatures and simple handlers, the interface for defining and
handling effects becomes small and easy to implement. Complicated effects can be
remapped to standard effects which can be handled generically, such as the
\texttt{State}, \texttt{Reader} or \texttt{Error} effect. The combination of
these features mean that effects are quick and easy to define. This makes it
feasible to implement systems with a much smaller finer effect granularity,
which in turn makes the effect reusable, easier to reason about and test.
An important contributing factor for the viability of this finely granular
approach is the improved performance characteristics as described in the last
section. Due to the linear scaling of freer monads finely granular effect
systems can be employed without having to expect massive performance penalty as
would be the case when using MTL.

\section{Alternative Effect Systems}

\label{sec:alternative-systems}

Effects are a common topic in the field of pure functional programming
languages. Particularly the recent years have enjoyed the development of several
new effect systems with a similar aim to the freer monads. Namely providing
extensible and performant effect systems.

\subsection{Handlers in Action}

\citeauthor{hia}~\cite{hia} Developed a an alternative system, building on free
monads, similar to the one from \citeauthor{freer}~\cite{freer}. The system,
called ``Handlers in Action'' (HIA), also leverages an augmented free monad, see
figure~\ref{fig:freer-monad}, the performance of which is enhanced by using the
continuation monad \texttt{Cont}, a construct which is very similar to the
\texttt{Codensity} monad as described before and fulfils a similar role here.
Whereas freer dealt with suites of effects, \texttt{ConsoleIO} for instance is
a suite consisting of two effects, \texttt{ReadLine} and \texttt{WriteLine},
``Handlers in Action'' defines every effect individually. Every effect thus
becomes its own top-level construct. Handler functions can choose which set of
effects to handle, as well as whether additional effects, which are delegated,
should be allowed or not. Handlers that allow additional effects are called
``shallow'' and the default case for freer. Handlers do not allow additional
effects are called ``deep'' handlers. Deep handlers can provide performance
benefits due to fewer dispatches.

To implement this in a user friendly way \citeauthor{hia} leverages the code
generation tool ``Template Haskell'' to generates type classes, constraints and
data structures for effects, handlers and functions using effects.

In this approach no open union is needed. Instead each effect is directly paired
with its handler via a \texttt{Handles} type class. Since each handler function
is member of a type class it will be a static top level value, thus reducing the
memory overhead because no additional \texttt{Union} value is needed as well as
improving the speed slightly since not every handler attempts to handle every
effect, but only the appropriate handler handles its own effect. However this
depends largely on whether or not the compiler inlines the instance
dictionaries.

In the performance evaluation in \cite{freer} the HIA system performed on-par or
better than freer. In fact for the simple example of the state monad it even
outperformed MTL. As far as I can tell this is due to the fact that, unlike
freer, HIA uses more builtin structures, like the continuation monad and type
classes. These structures are more transparent to the compiler, and have been
around for a long time, thus enjoy more dedicated optimisations. Type level
sequences and generic unions, as used by freer, are opaque to a large degree and
recent additions to Haskell and hence harder to optimise by the compiler.

In addition to Haskell HIA also provides implementations for Racket and OCaml.
Due to the less powerful type systems of these two languages the implementation
of the handlers is slightly different, however the semantics of the ensuing
program are similar. They use runtime tracking of handler functions in scope,
which is a more dynamic approach. An advantage of this system is that new
operations can be defined at runtime which simplifies certain tasks. One such
example (taken from the paper~\cite{hia}) is the representation of mutable
references by a set of dynamically created \texttt{Put} and \texttt{get}
operations.

\subsection{Algebraic Effects in Idris}

Certain of the features of the effect system depend on the language used to
implement it. The Idris~\cite{idris}\cite{idris-paper} programming language is
fully dependently typed and thus more powerful than Haskell. One result of this
is that effects can be even more finely disambiguated. Whereas in freer and
extensible effects several effects of a polymorphic type can be disambiguated in
the same computation, see section~\ref{sec:freer}, in the effects system
implemented by
\citeauthor{algebraic-effects-idris}~\cite{algebraic-effects-idris} in Idris,
called ``Effects'', even effects instantiated with the exact same types can be
disambiguated by means of adding labels to a particular effect. Subsequently the handlers
are selected based on the label, not the type alone. An example of how this
proves useful is if a computation were to carry two counters which are
incremented separately. Rather than having to redefine an effect the same
handler can be used and a label to disambiguate which counter is to be
incremented.

Another key difference in Idris is that effect handlers are defined via type
classes, similar to ``Handlers in Action'', as opposed to the user-defined
handler compositions of freer. In Idris case handlers are parameterised over a
context monad in addition to the effect itself. This allows it to be more
versatile and implement different handlers for different contexts. A type class
based approach can reduce the amount of code necessary to run effects, it does
however increase the amount of boilerplate necessary to run handlers in custom
contexts, such as when mocking for testing.\footnote{The reason is that in order
  to choose different handlers a newtype has to be instantiated as base monad
  and instances provided for every effect needed. Whereas in freer predefined
  handlers can simply be composed anew.}

The effect type in Idris is also more powerful when it comes to the type of
interactions that can be described by it. Effects are parameterised over a type
of input resource and a type of output resource. These are type level constructs
that live in the environment of the effect monad. An example from the paper by
Brady~\cite{algebraic-effects-idris} is a \texttt{FileIO} effect which allows
reading and writing to files. Rather than returning some kind of file handle
when opening the file and writing to the handle the \texttt{OpenFile} action
instead records the file in the computation environment including the mode of
opening. Neither reading nor writing takes a handle as argument, but instead
retrieves the resource from the environment automatically. Similarly close file
retrieves the ambient value but also removes the resource type from the
environment. As a result a closed file can never be written to or read from,
since the type checker detects that this resource was removed from the
environment at compile time. Similarly reading and writing cannot occur on files
opened with the incorrect mode, due to the mode also being recorded in the
environment of the effect. To open several files in the same computation the
aforementioned labels are used to disambiguate the effect targets.

A downside of the Idris approach is that the effect monad is no longer a monad
conforming to the \texttt{Monad} type class, because effects may change the the
environment resources and thus the type of the computation. As an alternative
Idris supports operator overloading to leverage the convenient \texttt{do}
notation regardless. Unlike previous approaches this effect system does not use
a simple augmented free monad but a much richer data structure that encodes
several other actions in addition to the two monadic actions \texttt{pure} and
$\bindOp$. This is due to the larger feature set of ``Effects'' compared to the
systems mentioned previously.

\subsection{PureScript}

One honourable mention at this point goes to the PureScript~\cite{purescript}
language. PureScript is a pure functional language, similar to Haskell, that
compiles to JavaScript. JavaScript has a lot of different built-in effects, such
as console I/O, DOM interaction, Websockets, Browser history modification, and
even more when NodeJS is used, adding process spawning, file interactions et
cetera. To deal with such myriad of effects, while avoiding the IO-Monad
problem, as it exists in Haskell, PureScript has built in support for a an
extensible effects monad (\texttt{Eff}).

An unfortunate issue in Haskell is that all effects dealing with some kind of
external API live in the monolithic \texttt{IO} monad. A signature \texttt{IO a}
does indicate that \emph{some} kind of effect occurs, but not which. It could be
reading a file, or launch missiles, an infamous example of how arbitrary effects
are from the Haskell community. PureScript addresses this by providing the
\texttt{Eff} monad as a built-in which is parameterised over an extensible
record of effects. The interface is very similar to the freer monad from the
previous section. The difference being that all effects in PureScripts
\texttt{Eff} are native effects. Each effect type is an untouchable empty native
type with no runtime representation and it carries no data. Similarly the
tagging of the imported native functions with effects is done arbitrarily by the
user and no alternative, ``pure'' implementations directly in PureScript are
possible.

However the use \texttt{Eff} as a built-in structure as well as using an
extensible record to track the composition of effect types means that all of the
code using \texttt{Eff} can be optimised to such a degree that no overhead
remains at runtime. Both effect resolution and $\bindOp$ inlining can be done at
compile time. Furthermore having built-in extensible records result in a concise
syntax.

\section{Conclusion}

This paper presents the conception and development of extensible effect systems.
It shows how a monad capable of handling extensible sets of effects can be
constructed by decoupling the effect from the monad using a free monad and
replacing concrete effects with an extensible union type. The implementation is
reviewed and optimised to attain competitive performance with transformers and a
minimal interface for effects is found resulting in several improvements over
current solutions:

\begin{itemize}
  \item Defining new effects involves minimal boilerplate. Necessary components
    are only an encoding data type and an interpreter.
  \item Computations using effects are constrained using the same member
    constraint regardless of effect, relieving the need for effect classes and
    instances.
  \item The effect monad is much more transparent, enabling combinators for
    effect relaying and generic handling.
\end{itemize}

Other, similar, solutions are examined and compared, which suggest that even
more powerful effect systems can be built upon these foundations and that
extensible effects may be a safer, more expressive alternative to monolithic
monads such as IO.


\bibliographystyle{ACM-Reference-Format}
\bibliography{bibliography}

\end{document}
