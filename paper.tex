\documentclass[sigconf]{acmart}

\usepackage{booktabs} % For formal tables
\usepackage[utf8]{inputenc}
\usepackage{listings}


\newcommand{\defineMonad}[2]{\newcommand{#1}{\texttt{#2}}}

\defineMonad{\ReaderM}{Reader}
\defineMonad{\WriterM}{Writer}
\defineMonad{\IOM}{IO}
\defineMonad{\StateM}{State}
\defineMonad{\ExceptM}{Except}
\defineMonad{\RedisM}{Redis}
\defineMonad{\HandlerForM}{HandlerFor}
\defineMonad{\EitherM}{Either}
\defineMonad{\IdentityM}{Identity}

\defineMonad{\ExceptT}{ExceptT}
\defineMonad{\ReaderT}{ReaderT}
\defineMonad{\StateT}{StateT}
\defineMonad{\WriterT}{WriterT}


\newcommand{\bindOp}{\gg=}
\newcommand{\bindIgnOp}{\gg}

\newcommand{\Functor}{\texttt{Functor}}
\newcommand{\Monad}{\texttt{Monad}}


\lstset{%
  language=Haskell,
  basicstyle=\footnotesize,
  literate=
  {->}{{$\rightarrow$}}1 {=>}{{$\Rightarrow$}}1 {>>=}{{$\bindOp$}}2 {>>}{{$\bindIgnOp$}}1
}

% Copyright
%\setcopyright{none}
%\setcopyright{acmcopyright}
%\setcopyright{acmlicensed}
\setcopyright{rightsretained}
%\setcopyright{usgov}
%\setcopyright{usgovmixed}
%\setcopyright{cagov}
%\setcopyright{cagovmixed}

\begin{document}
\title{History and Evolution of Freedom for Monads}

\author{Justus Adam}
\affiliation{}
\email{justus.adam@tu-dresden.de}


\begin{abstract}
Side effects are the sore spot for pure languages. The mathematical
beauty of referential transparency has to be broken to allow crude and
unpredictable outside interactions, such as network and filesystem
accesses, that are necessary to write useful programs.

Fortunately with the Monad an abstraction was discovered that allowed
safe and transparent, sequential interaction with an arbitrary ambient
environment.

Yet it too is flawed. Versatile though the Monad is, composing
effects in a reusable fashion proved difficult and has performance
implications which increase in severity as the granularity gets finer.

With the free monad we now finally not only obtain a way to compose
arbitrary effects together in an efficient way but can also easily
implement effects via transformation into other effects and even
implement the same effect in different ways depending on needs and
constraints of its application.

This paper provides an overview of the origins of the Monad, how its
shortcomings lead to the invention of transformers, then the free
monad, and eventually the freer monad, how free and freer monads work
and how they can be used to realise our collective dream of pure,
efficient, composable and versatile ambient interactions.
\end{abstract}

\maketitle

\section{Introduction}

\label{sec:introduction}

\section{Monads}

\label{sec:transformers}

Monads are an abstraction used to implement effects on an ambient environment in
languages such as Haskell and Idris~\cite{wadler-monads}. Inherently monads must
only satisfy and interface of two function, \texttt{pure :: Monad m
  $\Rightarrow$ a $\rightarrow$ m a} to embed any value in the monad, and
\texttt{($\bindOp$) :: Monad m $\Rightarrow$ m a $\rightarrow$ (a $\rightarrow$
  m b) $\rightarrow$ m b}, also called \emph{bind}, to sequence two monadic
computations. While this captures the sequential nature of the monad, it does
not provide any way to extract a value, or combine one monad with another,
because the chained operation must live in the same monad as the value it is
chained to.


Purity, often also called referential transparency, is the property of
a function to always produce the same result when called with the same
arguments. It is a desirable property as it makes it considerably
easier to reason about the behaviour of a program, particularly with
respects to refactoring. But it also allows optimisations, such as
common subexpression elimination and memoization.

However, many tasks that we wish programs to perform cannot be
expressed in terms of pure functions as they entail an interaction
with the world outside of the program, such as accessing a database or
querying the file system. Even inside of the program it is often
useful to define some ambient environment in which certain tasks are
performed, from this object oriented programming was conceived where
each function, or method, carries around the implicit environment of
an object.

To perform the aforementioned tasks many programming languages opt to sacrifice
purity and allow arbitrary side effects in any function. The designers of the
Haskell programming language however found a way to enable side effects in
programs without sacrificing purity. \textbf{Monads}, in the Haskell sense, are
a class of types that describe some sort of environment, which can be interacted
with. The concrete interactions a particular monad allows differ but they have a
common notion of sequentiality, encoded with the \textbf{bind} operator
(\texttt{($\bindOp$) :: Monad m $\Rightarrow$ m a $\rightarrow$ (a $\rightarrow$
  m b) $\rightarrow$ m b}), which connects an action performed with the monad
and a continuation that requires the result of this interaction, and the ability
to embed pure values into the monad (\texttt{pure :: Monad m $\Rightarrow$ a
  $\rightarrow$ m a}).

Of particular importance is the sequentiality of actions that is enforced by
$\bindOp$. Since the left hand argument is a continuation the right hand action
must be performed before the program can advance. This structure enforces an
order to the execution of side effects whether or not these actions produce
actual results, such as writing to a table in a database.

The Monad proved to be a very successful concept in describing
sequential interactions with ambient environments and thus many
different monads had soon been developed. Monads for interacting with
the system (\texttt{IO}\footnote{Part of the \texttt{base} library}),
for interacting with databases (\RedisM{}~\footnote{Part of the
  \texttt{hedis}\cite{hedis} package for interacting with the Redis
  Key-Value-Store}) or handling web requests in a server
(\HandlerForM{}\footnote{Part of the Yesod~\cite{yesod} web
  framework}) as well as monads that defined environments embedded in
the program such as \texttt{Writer}, which collects outputs,
\ReaderM{}, which adds a static environment and
\ExceptM{}\footnote{\ExceptM{} does not actually exist. Only the
  transformer \ExceptT{} does as \ExceptM{} would be the same as
  \EitherM{}. \ExceptM{} is only used here to make the connection to
  its transformer more obvious}, which adds throwing and handling of
user defined errors.

Though each of these monads are well suited for interacting with the
various environments they describe, they are rather unwieldy, if not
impossible to use when we wish to interleave their effects. We may
desire to read some data from the \RedisM{} database, perform some
network \IOM{} afterwards and finally send some computed result as
part of being a \HandlerForM{}, all the while tracking potential
\ExceptM{} errors.

This however is not possible with a simple monad. Most of these monads
are opaque types, results from which can only be obtained by
performing complex set-up and tear-down operations, and some, like the
\HandlerForM{} monad offer no way for the user to extract the pure
data directly. This makes sense of course as the creation of this data
entails certain opaque interactions with the environment from which it
cannot be easily untangled. Though the power of the monad lies
precisely in leaving those interactions opaque it poses a challenge
when trying to achieve composability and interleaving of effects.

The next section offers an overview of the current most popular
solution for achieving composability: effect classes and monad
transformers.

\section{Transformers and Classes}

\label{sec:transformers}

The need for ambient effects is different for each program. Whereas effect
monads like \IOM{} and \HandlerForM{} are absolutely required, as they provide
the only facility to interact with certain resources, additional ``convenience''
monads such as \ReaderM{}, \WriterM{}, \StateM{}\footnote{Similar to the
  \ReaderM{} monad but the environment is not static and can be altered during
  the computation} and \ExceptM{} are ``optional'' and provide easier interfaces
for describing certain types of computation.

For each domain there is a different set of effect combinations particularly
suited to describe it. When terminal monads are used, like those mentioned in
the last paragraph, an entirely new monad would have to be implemented for every
combination of effects we desire. Alternatively one large monad, which includes
\textbf{all} effects could be used, but then each effect would necessarily have
to be handled when the monad is run, regardless of whether the computation
actually uses it. And how would the domain specific effects like \IOM{} and
\HandlerForM{} be added to this monolithic monstrosity?

Clearly neither of these approaches is well suited to solve the need for
interleaved effects. A system is needed whereby larger, more complicated, monads
can be composed of smaller, elemental ones. Ideally this could be done
dynamically, as some sections of the code may require additional effects to be
added, like adding a \WriterM{} to accumulate the results of some embedded
computation.

The most popular solution currently is the one implemented by the
\texttt{mtl}\cite{mtl} and \texttt{transformers}\cite{transformers} library and
it revolves around using stacks of so called \textbf{monad transformers} and
\textbf{monad classes}. The approach was inspired by a paper from Mark P.
Jones~\cite{transformer-inspiration}. The idea is that, rather than defining
terminal monads, such as \ReaderM{}, the defined monad is parameterized by an
additional \emph{inner monad} to which other effects can be delegated. The
resulting structure is called a \emph{transformer} as it transforms a given
monad by adding an additional type of effect. Transformers are typically named
after their effect with a capital ``T'' prepended. Thus the transformer for the
\ReaderM{} monad would be \ReaderT{}.

This process of wrapping a monad with a transformer can be done arbitrarily
often as the result of wrapping a monad with a transformer again produces a
monad which can become the inner monad to another transformer. The resulting
structure is often also referred to as a transformer \emph{stack} that is
terminated at the last level by some terminal monad which cannot be used as a
transformer, such as \IOM{} or \HandlerForM{}.

When the computation is executed the layers of transformers are removed
individually, the outermost layer always first until a computation in the base
monad remains. Transformer stacks can even be used to describe pure computations
by choosing \IdentityM{} as the inner monad, which resolves to a pure value.
Additionally, during a computation, new (outer) layers can be dynamically added
and removed from the monad. As such for instance a subcomputation can be run
with an additional \WriterM{} effect, that is then separately unwrapped to
obtain the written results.

The characteristic action of a transformer, as captured by the
\texttt{MonadTrans} class, is \texttt{lift :: (MonadTrans n, Monad m)
  $\Rightarrow$ m a $\rightarrow$ n m a} which can be seen as embedding an
action of the inner monad in the outer monad, or, alternatively, as delegating
(lifting) an effect to be handled by the inner monad. Unfortunately this lifting
operation poses a problem when the transformer stack get larger. Performing an
effect which is far up the stack will require several lifting operations to
reach the transformer that can actually satisfy it.

To deal with this issue another concept is used in the form of \textbf{monad
  classes}. Rather than defining effects on a transformer directly a type class
is defined with the interface for the effect in question, see for instance the
effect as provided by the \StateM{} monad in
figure~\ref{fig:monad-class-example}. Transformers responsible for handling the
effect are then made instances of this class and implement it accordingly.
Additionally all other transformers are also made instances of this class but
use \texttt{lift} to delegate the effect up the stack. Thus when such an effect
is used it automatically propagates up the stack until it reaches the
transformer capable of handling it (should one exist\footnote{Should no capable
  handler exist the type checker will report a class instance resolution error.}).

\begin{figure}
  \begin{lstlisting}
    class MonadState s m | m -> s where
      put :: s -> m ()
      get :: m s
  \end{lstlisting}
  \caption{The monad class for the \StateM{} effect}
  \label{fig:monad-class-example}
\end{figure}

An additional advantage of monad classes is that effectful computations can be
defined solely in terms of the set of effects it uses, rather than the concrete
monad it operates in. As a result they can be reused in entirely different monad
stacks without modification of the code so long as it satisfies the same effect
interface. An example of this can be seen in figure~\ref{fig:monad-class-poly}.
Whereas the second function, \texttt{writeState1}, can only be used in the
explicit \StateT{}, \WriterT{} stack, the first function, \texttt{writeState},
could also be used in a stack where the two transformers are reversed, or if it
were wrapped by additional transformers.

\begin{figure}
  \begin{lstlisting}
    writeState :: ( MonadState s m
                  , MonadWriter s m
                  , Monoid s )
               => m ()
    writeState = get >>= tell >> put mempty

    writeState1 :: ( Monad m
                   , Monoid s )
                => StateT s (WriterT s m) ()
    writeState = get >>= tell >> put mempty
  \end{lstlisting}
  \caption{Polymorphic and explicit effect signatures}
  \label{fig:monad-class-poly}
\end{figure}

While this provides a clean interface for interacting with the various effects
in our stack, using the class-based effect functions like \texttt{get}, it also
poses some issues. In the interface definition for the \StateM{} effect, for
instance, the type of state, which can be handled, is fixed to the monad, as
expressed by the \texttt{m $\rightarrow$ s} functional constraint. This means a
stack can only ever handle the state effect for \emph{one single type of state
  $s$}. Adding additional \StateT{} transformer will shadow any \StateM{} effect
of its inner monad and render it unreachable using the \texttt{MonadState}
class~\footnote{Theoretically it is still reachable using a combination of
  explicit \texttt{lift} and the effect function but using \texttt{lift}
  reintroduces the issues from before that prompted the addition of monad
  classes in the first place.}. This becomes particularly troublesome when
libraries are used that implement certain interactions in terms of effects on a
specific type, like \ReaderM{} on a specific type \texttt{t} but the monad it is
supposed to run in is already a \ReaderM{} for some \emph{other} type. In these
cases it becomes rather tedious to wrap and unwrap every library interaction
with another \ReaderT{} to shadow ones own \ReaderM{} effect. This problem can
be eased using lenses but requires that the library be defined both in terms of
lenses as well as monad classes.

Monad class instances themselves also pose issues. In general it is beneficial
to define custom effects in terms of new monad classes and transformers. This
makes these effects extensible, composable and reusable, properties which are
desired, particularly in large projects which are frequently subjected to
refactoring and expansion.

However defining you own transformers is tedious. Not only does one have to
implement the transformer itself but additionally provide a monad class instance
for every \emph{type of effect} one could possibly combine with the new
transformer. Additionally a new class for the custom effect as well es instances
of this class for every \emph{existing transformer} is necessary to achieve the
maximum amount of composability. In case of GHC some of these can be
automatically derived by the compiler, but only when \texttt{newtype} wrapping
is used. When the granularity of the custom effects is decreased, that is they
are broken into smaller and smaller pieces to achieve modularity, more and more
of such definitions and instances are required, bloating the code.\footnote{One
  particularly good example of this I believe is the \texttt{monad-logger}
  library, paticularly this module
  https://github.com/snoyberg/monad-logger/blob/master/Control/Monad/Logger.hs
  where they even resort to the use of preprocessor macros to reduce the code
  duplication.}

A truly extensible system should not require changes to existing effects in
order to incorporate new ones. Ideally we would only have to define our effect
interface and a means to perform them and the system would implicitly know how
to combine it all other effects.

The next section introduces a first such system by first removing the
boilerplate of defining monads in the first place and then moves on to extend
this into a composable system of effects.

\section{Freedom to the Rescue}

\label{sec:free}

In Section~\ref{sec:monads} the monad was introduced as a way to
sequence effectful computations. Then various types of monads were
found, encoding different effects.

\subsection{Decoupling Monad and Effect}

\label{sec:introducing-free}

One of the most important discoveries \textbf{CITATION} was that the
effect and its sequential nature can be separated and the latter
encoded with a generic data structure. The essence of the monad,
$\bindOp$ and \texttt{pure} is captured using the so called
\textbf{free monad} (see figure~\ref{fig:free-monad}). This generic
monad is parameterized by an effect type \texttt{f} which itself only
needs to be a \Functor{}. Functors are much simpler structures than
monads. They have no notion of sequentiality and continuations which
makes them easier to implement. A Functor only needs to allow the
application of a function to a contained value as illustrated by the
type class \Functor{} displayed in figure~\ref{fig:functor-class}.

\begin{figure}
  \begin{lstlisting}
    class Functor f where
      fmap :: (a -> b) -> f a -> f b
  \end{lstlisting}
  \caption{The \Functor{} interface}
  \label{fig:functor-class}
\end{figure}
\begin{figure}
  \lstinputlisting[firstline=8]{Listings/Free.hs}
  \caption{The simple free monad}
  \label{fig:free-monad}
\end{figure}
\begin{figure}
  \begin{lstlisting}
    join :: Monad m => m (m a) -> m a
    join ma = ma >>= id

    (>>=) :: Monad m => m a -> (a -> m b) -> m b
    ma >>= cont = join $ fmap cont ma
  \end{lstlisting}
  \caption{The equivalence of $\bindOp$ and \texttt{join}}
  \label{fig:join-bind-equivalence}
\end{figure}

The two constructors for the free monad represent the two
characteristic functions of the \Monad{} in category theory,
\texttt{pure} and \texttt{join}, the latter represented by the
\texttt{Impure} constructor. The \texttt{join :: Monad m => m (m a) ->
  m a} function allows flattening of two stacked, identical monadic
contexts. It, in conjuction with \texttt{pure} is sufficient to
characterize a monad the same way that $\bindOp$ does since either one
can be implemented in terms of the other, see
figure~\ref{fig:join-bind-equivalence}.

In case of the \texttt{Impure} constructor the continuation
(\texttt{Free f a}) is contained in a layer of the effect functor
\texttt{f} meaning that proceeding with the computations requires
execution of one instance of the effect. Only then is an
interpretation of the inner (later) computation possible, including
the effects used later in the computation. This ensures effects are
executed in the same order that they are used in, in the program.

Using this data structure removes the need define a \Monad{} instance
for the effect type \texttt{f}, as the free monad is a true \Monad{}
for \emph{any} \Functor{} \texttt{f}, see
figure~\ref{fig:free-monad}, this also means all the
useful combinators defined for monads are available to an effect
system defined like this. Effects are entirely delegated to the
functor \texttt{f} allowing completely independent and different kinds
of effect systems to be implemented using \texttt{Free}.

To run a computation expressed in a free monad effect system is done
by a function typically called an \emph{interpreter}. This interpreter
implements how the effects should be handled in a certain
environment. One additional advantage of using an interpreter is that
effects can be handled differently depending on the concrete
interpreter. They can even run in different environments entirely
without requiring a change in the computations code.

\begin{figure}
  \lstinputlisting[firstline=7]{Listings/ConsoleIO.hs}
   \caption{Implementing a simple console I/O interaction using
    \texttt{Free}}
  \label{fig:console-io-example}
\end{figure}

An example can be seen in figure~\ref{fig:console-io-example} where a
simple console interaction is implemented using the free monad. Here
two very different interpreters are provided. One where the
interpretation happens in the IO monad and the two requests
\texttt{ReadLine} and \texttt{WriteLine} are directly delegated to the
usual I/O functions and a second one where the requests are served
from a list of input strings and the written lines are again recorded
in a list. This power of the free monad effects to be interpreted in
different ways makes it ideal to implement mock interpreters for
testing functions as well as making code more reusable in different
scenarios. Transformers for instance do not offer the same degree of
flexibility.

\subsection{Composing Effects}

\label{sec:simple-effect-composition}

Decoupling the monad from its effect enables the composition of
effects using functor composition. Two effect systems implemented
using the free monad can be composed into an effect system
implementing both types of effects by composing the two effect
functors into a single one and using the result functor to
parameterize the free monad. When handling the effect the two functors
are decomposed again and the individual effects are dispatched to
corresponding handlers.

\begin{figure}
  \lstinputlisting[firstline=5,lastline=9]{Listings/SimpleOpenUnion.hs}
  \caption{Then sum functor \texttt{:+:}}
  \label{fig:sum-functor}
\end{figure}

The simplest way to combine effects is the sum functor (\texttt{:+:}) as seen in
figure \ref{fig:sum-functor}. If combines two functors \texttt{f} and
\texttt{g}, dispatching requests accordingly. We can use this simple union of
two types to express functors that combine arbitrary types by nesting
\texttt{:+:} and add projection (\texttt{prj}) and injection (\texttt{inj}) via
a subtype relation class \texttt{:<:}, see figure \ref{fig:dispatch-class}. This
way we form a so called \emph{Open Union} of functors. The sum functor and the
open union is a technique developed by Wouter
Swierstra~\cite{data-types-a-la-carte} and based on the \texttt{OR} type from
Liang, Hudak and Jones~\cite{monad-transformers}, lifted to the higher kinded
functors.

\begin{figure}
  \lstinputlisting[firstline=13]{Listings/SimpleOpenUnion.hs}
  \caption{Dispatch classes for effects in a simple open union}
  \label{fig:dispatch-class}
\end{figure}

Using an open union like this as the base functor for the free monad allows us
to parameterize functions over the types of effects they use without directly
describing the structure of the functor and without the need for additional type
classes and class instances for every new effect. An extensible version of the
\texttt{readLine} and \texttt{writeLine} functions from the earlier example can
be seen in figure \ref{fig:extensible-console-effect}. Whereas the earlier
version from figure \ref{fig:console-io-example} constrained us to use the
explicit type of \texttt{Free Console}, the extensible version allows any
version of \texttt{Free}, so long as the effect functor contains the
\texttt{Console} effect. Because of the \texttt{f :<: f} instance of the
dispatch class \texttt{:+:} this also includes \texttt{Console} itself.

\begin{figure}
  \lstinputlisting[firstline=8]{Listings/ExtensibleConsole.hs}
  \caption{An extensible version of the console interations}
  \label{fig:extensible-console-effect}
\end{figure}

As the functions are now independent of the concrete type of underlying functor
we can combine effects freely. In figure \ref{fig:time-effect} for instance we
can see another effect which retrieves the current time. A function that wishes
to use both effects can now do so, again without having to specify any
particular structure of the functor but simply gaining a constraint, see figure
\ref{fig:combined-console-time-effect}.

\begin{figure}
  \lstinputlisting[firstline=9]{Listings/TimeM.hs}
  \caption{An effect for retrieving the current time}
  \label{fig:time-effect}
\end{figure}

\begin{figure}
  \lstinputlisting{Listings/CombinedTimeConsole.hs}
  \caption{A function using both \texttt{Console} and \texttt{Time}}
  \label{fig:combined-console-time-effect}
\end{figure}

In addition the combination of effects we can also implement an
\texttt{interpose} combinator which can be used to alter effects, see
figure \ref{fig:interpose-combinator}. This can be used for instance to attach a
timestamp to each message sent to the console by use of the \texttt{Time} effect
or implement input filtering, see figure \ref{fig:attach-timestamp}.

\begin{figure}
  \lstinputlisting[firstline=13,lastline=22]{Listings/Interpose.hs}
  \caption{The interpose combinator}
  \label{fig:interpose-combinator}
\end{figure}

\begin{figure}
  \lstinputlisting[firstline=24]{Listings/Interpose.hs}
  \caption{Two examples for using the interpose combinator}
  \label{fig:attach-timestamp}
\end{figure}

\subsection{Handling Effects}

While it is not necessary to know about the concrete structure of the effect
functor to use its effects it is necessary to handle them. The nesting of sum
functors forms a linked list and effects have to interpreted starting from the
head of the list. Various combinators are available to handle these effects. In
figure~\ref{fig:interpreting} the \texttt{interpret} combinator is shown which
interprets an effect in terms of other effects present in the list. Similarly it
shows the signatures of other combinators. Some effects, like the
\texttt{Reader} effect shown in figure \ref{fig:reader-effect} can be
interpreted directly. Others may need to be delegated. The idea is that we
either handle effects or delegate until we have unrolled the whole effect list
and are left with some base monad (often \texttt{IO}) which we run directly
using the \texttt{runM} or with no effect at all and we can simply return the
pure value with \texttt{run}, both can be seen in figure \ref{fig:interpreting}.

\begin{figure}
  \lstinputlisting[firstline=10]{Listings/Interpret.hs}
  \caption{Combinators for interpreting effects}
  \label{fig:interpreting}
\end{figure}

It may not seem immediately obvious, but this retains the same flexibility we
saw earlier in the \texttt{Free} monad whereby effects could be interpreted
differently based on context. In this case it is even easier to do this. For
instance in the example of handling the console effect our computation will have
a type of \texttt{(Console :<: sum, Functor sum) => Free sum a}. If this is to
be interpreted in the \texttt{IO} monad \texttt{sum} can be instantiated to be
\texttt{Console :+: IO} and interpreted with the \texttt{interpret} function.
All console effects are delegated using the \texttt{liftIO} helper and the two
handlers \texttt{putStrLn} and \texttt{getLine} which were also used in the
earlier example in figure~\ref{fig:console-io-example}. A computation of type
\texttt{Free IO a} remains, which we can be run using the \texttt{runM}
function. If instead the interpretation should happen in a pure context, similar
to the example before, \texttt{sum} can instead be instantiated to
\texttt{Console :+: Identity}. Using the \texttt{reinterpret} function, the
\texttt{Console} effect is turned into a \texttt{State ([String], [String])}
effect. \texttt{State} is a standard effect and its implementation omitted here.
It can be run using the \texttt{runState :: s -> Free (State s :+: sum) a ->
  Free sum (a, s)} handler. After interpreting \texttt{State} only \texttt{Free
  Identity (a, s)} is left, which fits the type signature for \texttt{run}.
Using \texttt{run} the final type will be \texttt{(a, ([String], [String]))},
which is the result of our computation, as well as the remaining, unread lines
of input and the lines written to output. Code for this example can be viewed in
figure~\ref{fig:run-console}. In short we retain the flexibility of
context-dependent interpretation because our computation is unaware of the
underlying effect layers and thus we are free to instantiate them as necessary.

\begin{figure}
  \lstinputlisting[firstline=42]{Listings/RunConsole.hs}
  \caption{Interpreting the extensible console effect}
  \label{fig:run-console}
\end{figure}

\subsection{Summary}

Using a free monad we can decouple the computation using an effect from the way
that effect is handled. Effects themselves need only concern themselves with
encoding the effect in a functor and the free monad provides the sequentiality
``for free''. By using an open union the functions using the effect can be
implemented unaware of the concrete structure of the underlying effect functor.
This means that new effects can be added in a modular fashion without the need
to change the functions using them.

Interpreting the various effects is highly flexible and can be changed depending
on the context of the computation, allowing easy mocking for instance.
Furthermore using the open union allows the interpretation of effects in terms
of other effects that can be handled easier or are generic effect types.

What was set out to do has been achieved, a highly modular, easy to extend and
flexibly interpretable system for defining effectful computations.

However an issue remains with regards to efficiency. Particularly the
implementation of an open union using a linked list of functors means that some
part of the list has to be traversed on every \texttt{fmap}, which is used
ubiquitously in the implementation. This also means that the performance
degrades as the number of different effect types grows. Smaller effect
granularity, which leads to greater reusability and makes it easier to reason
about effects, is punished by degrading performance, an undesirable effect.
Furthermore the $\bindOp{}$ implementation also recurses into the continuation on
every bind, leading to bad asymptotic performance. The next section will explore
an alternative way to implement the free monad and open union to improve the
performance.


\bibliographystyle{ACM-Reference-Format}
\bibliography{bibliography}

\end{document}
